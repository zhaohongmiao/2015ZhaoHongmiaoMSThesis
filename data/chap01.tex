
%%% Local Variables:
%%% mode: latex
%%% TeX-master: t
%%% End:

\chapter{绪论}
\label{cha1}

\section{研究背景及意义}
\label{1_1}

人类及其他灵长类动物在面对复杂场景时能够简单快速的找到自己感兴趣或比较显著的区域,并对其进行优先处理,这个过程叫做视觉注意。视觉注意机制是视觉系统本身的一种属性,是为适应环境而进化的产物,在视觉信息处理过程中有着十分重要的意义。每天人类及其他灵长类动物的眼睛都要接受海量的信息,但是视觉信息处理能力是有限的,视觉系统无法将图像中所有的信息都进行处理,并且我们所关注的内容只占一小部分。所以,通过视觉注意这种机制,我们可以对感兴趣的信息进行选择性处理,并且给予不同的处理级别,从而减小了信息处理的复杂性,节省了时间,在很大程度上提高了视觉系统的信息处理效率。

视觉注意根据心理和生理机制可以分为两种,一种是自底向上的视觉注意,另一种是自顶向下的视觉注意。自底向上的视觉注意不受意识控制,独立于具体任务,靠数据进行驱动,处理信息时速度相对比较快,比如你的面前有一堆绿色的苹果,其中只有一个苹果是红色的,当这些苹果映入你眼帘时,你会不由自主的只注视那只红色的苹果。自顶向下的视觉注意取决大脑高级皮层先验信息的反馈,因此受意识支配,处理信息的速度相对较慢,比如还是前面的例子,这一堆绿苹果中只有一个红苹果,但是也包含一个烂了的苹果,现在你准备把这个烂苹果从苹果堆里挑出来,因此当你面对这一堆苹果的时候,你就会将注意力集中在坏掉的苹果上,而不是红色或绿色的苹果。

视觉注意机制是一个多学科多交叉的领域,自研究以来得到了许多学科研究者的关注,如:神经科学、生理学、心理学以及计算机科学等。其中,将视觉注意引入到计算机领域称为显著性检测,即检测出图像的显著性信息,忽略冗余信息,使图像处理的结果更加符合人类的视觉感受。随着计算机科学技术的发展,视觉注意计算模型逐渐成为计算机视觉及图像处理研究者感兴趣的热点,并越来越多的应用到计算机视觉领域,如图像分割、目标识别、图像重定向及视频压缩等。

空间域视觉注意仿生模型较好的模拟了人类的视觉注意机制,但是由于过多地模拟这种生理机制使得计算量相对较大,比较耗时。基于信息论和统计等显著性检测模型需要估测概率密度或建立概率模型,引入大量参数,并且计算效率比较低,因此,不适于工程性应用。频域处理具有简单、高效及参数设置少的特点,可以对图像进行实时性处理,所以自底向上的频域视觉注意模型对于显著性检测是很好的选择。

\section{国内外研究现状}
\label{1_2}

视觉注意在计算机领域的研究被称为显著性检测。目前,有关显著性检测的研究可以大致分为两类:注视焦点预测和显著性目标检测\cite{ChengMingMingCVPR2014BING,LiYinCVPR2014Secrets}。注视焦点预测旨在通过计算显著图来模拟人眼观测点。在Koch与Ullman\cite{Koch1987Shifts}提出的特征融合理论和神经生物学框架的启发下,Itti等人\cite{IttiTPAMI1998Model}建立了第一个自底向上的显著性检测计算模型。该模型利用线性滤波的方法提取图像的颜色、亮度和方向等特征得到多尺度的特征图高斯金字塔,通过中央周围差和归一化算子获得特征显著图,然后将这些特征显著图通过线性融合机制融合成显著图,采用赢者全取的策略支配视觉焦点的转移。该模型较为完整的模拟了视觉注意机制,在计算机显著性检测领域具有里程碑的作用。注视焦点模型获得了很大的改进和发展\cite{JuddMIT2012Benchmark,BorjiTIP2013Quantitative,BorjiICCV2013Analysis,BorjiTPAMI2013State,VigCVPR2014Large},但是预测的结果往往趋向于突显边缘和角点等纹理较密集的区域,而非整个目标,因此,该种模型应用性不高\cite{ChengMingMingCVPR2014BING}。

显著目标检测是用来检测给定场景中最显著、最能引起人注意的整个物体,并将它完整的分割出来\cite{BorjiECCV2012Salient,LiYinCVPR2014Secrets}。Liu等人\cite{LiuTieCVPR2007Learning}通过条件随机场将局部、区域和全局的显著目标特征融合起来。Achanta等人\cite{AchantaCVPR2009Frequency}通过调频的方法,利用颜色和亮度信息进行显著区域检测。Cheng等人\cite{ChengMingMingCVPR2011Global}提出了基于区域对比度的显著性检测方法。最近,更多的研究方法\cite{BorjiECCV2012Salient}关注于如何使得显著性检测结果更精确、更鲁棒,如基于滤波器方程的方法\cite{PerazziCVPR2012filters}、判别区域特征融合的方法\cite{JiangHuaizuCVPR2013Discriminative}、基于图的流型排序\cite{YangChuanCVPR2013Manifold}、分层的方法\cite{YanQiongCVPR2013Hierarchical}和子模框架\cite{JiangZhuolinCVPR2013Submodular},另外还有统计纹理特征的方法\cite{ScharfenbergerCVPR2013Statistical}、有效数据表达\cite{ChengMingMingICCV2013Efficient}、基于上下文超图建模\cite{LiXiICCV2013Contextual}、布尔图的方法\cite{ZhangJianmingICCV2013Boolean}、基于马尔可夫链的方法\cite{JiangBowenICCV2013Markov}、稀疏性表达的方法\cite{LiXiaohuiICCV2013Dense}、偏微分方程学习的方法\cite{LiuRishengCVPR2014Adaptive}以及光场的方法\cite{LiNianyiCVPR2014Light}等等。显著目标检测强调检测出整个显著物体,因此它在计算机视觉领域具有广泛的应用\cite{BorjiTPAMI2013State},比如图像场景分析、内容感知、图像编辑、图像视频压缩等。尽管显著目标检测的精度越来越高,但在处理的过程中特征选择越来越多、算法越来越复杂,使得计算量越来越大,造成计算效率越来越低,因此,不便于进行实时性处理。

为了更为简单、快速、有效并且不依赖于分类或其他先验知识,频域显著性检测吸引了越来越多人的研究。Hou和Zhang首先将显著性检测引入到频域中,提出了谱剩余的显著性检测算法(SR)\cite{HouXiaodiCVPR2007Residual}。经过分析,Guo等人\cite{GuoChenleiCVPR2008Spatio}认为去掉幅度谱,只保留相位谱就可以恢复出图像的显著图而无需利用剩余谱,得出的结果跟SR算法的几乎一样。他们还将傅里叶变换扩展为四元数傅里叶变换,考虑到了图像的颜色、亮度等信息,将它们结合在一个变换中,提出了相位谱四元数傅里叶变换(PQFT)。之后,频域显著性模型又得到了更多的扩展,如脉冲主成分分析的脉冲离散余弦变换模型(PCA)\cite{YuICDL2009Spatio}和频域分解归一化模型(PDN)\cite{BianCognNeurodyn2010Visual}。以上方法虽然取得了一定的效果,但是他们仅仅可以检测出边缘、纹理复杂的较小区域,对于目标较大的区域检测效果并不理想。直到2013年,Li等人提出了基于谱尺度空间的显著性检测算法(HFT)\cite{LiJianTPAMI2013Scale}解决了这个问题,该算法将背景看作重复模式,对应于频域幅度谱的尖刺,通过不同尺度的高斯滤波器对幅度谱进行平滑,达到抑制背景、突出前景显著目标的目的。

\section{本文主要工作及安排}
\label{1_3}

本文主要对自底向上的频域显著性检测算法进行了研究,总结了国际上现有的频域显著性检测处理方法,并提出了基于自适应幅度谱分析的谱显著性检测算法。本文的主要安排如下:

第一章为绪论部分,主要对频域显著性检测算法的背景、意义及国内外研究现状进行介绍,并对全文的主要工作及安排进行了说明。

第二章介绍基于频域处理的显著性检测。详细介绍了算法的设计框架和流程,包括图像预处理、时频变换、谱处理、图像后续处理等,对每一个环节进行详细的分析和归纳。

第三章将国内外频域显著性检测方法进行整理和总结,完成对频域显著性检测经典算法的综述。

第四章针对频域显著区域检测问题设计了算法,详细介绍了算法的设计框架,并结合实际处理结果说明了方法的有效性,并通过在大量图像数据集上的实验证明了算法的优越性。

第五章对全文的工作进行了总结,并指出工作中存在的问题对以后的工作提出了展望。

\section{本章小结}
\label{1_4}

本章主要介绍了频域显著性检测的背景和意义,同时分析了国内外研究的现状,并对全文的工作和安排进行了介绍。

