\chapter{频域显著性检测模型}
\label{cha3}

在上一个章,我们主要对频域显著性检测的一般处理步骤进行了详细的说明,对图像频率谱进行了简单的分析。本章将对国内外典型的频域显著性检测模型进行详细的介绍,尤其是对图像转换到频域进行的谱处理进行具体分析。
%=============================================================================================================
\section{谱剩余显著性检测模型(SR)}
\label{3_1}

2006年,侯晓迪首次提出了一种既简单又快速的频域显著性检测算法---谱剩余算法(SR)~\cite{HouXiaodiCVPR2007Residual}。SR方法从信息论\cite{Barlow1961Possible}的观点认为在一幅图像或一个场景中所有的信息被分做两部分,一部分是图像的新颖部分代表显著性信息,另一部分是图像的冗余部分代表非显著性信息或先验信息,即:
\begin{equation}
H(Image)=H(Innovation)+H(Prior Knowledge)
\label{式3_1}
\end{equation}
根据有效编码理论,图像的冗余信息应该被抑制掉而保留图像的新颖部分。对于一幅图像如何去将这两种信息进行区分呢? 直观的理解认为,属于新颖部分像素的统计特性低于冗余部分的像素。自然图像统计的不变特性中,尺度不变性\cite{Ruderman1994Statistics,Srivastava2003Advances}被认为是最著名最广泛的一种属性,又称作$1/f$法则,其表达式如下:
\begin{equation}
E\{A(f)\}\propto 1/f
\label{式3_2}
\end{equation}
$A(f)$表示图像进行傅里叶变换后的幅度谱,$E\{\}$表示数学期望。它说明自然图像傅里叶变换的幅度谱是有规律的。图~\ref{图3_1}显示了不同图像的对数幅度谱,从图中可以看出不同图像所对应的曲线具有相似的形状,这种相似性说明了自然图像的冗余所在。如果对多幅图像的对数幅度谱进行平均处理,从图~\ref{图3_2}可以发现,处理后的曲线比较平滑,这说明了均值对数幅度谱具有局部线性的特征\cite{HouXiaodiCVPR2007Residual}。
\begin{figure}[!t] %插图
\centering
\includegraphics[width=0.8\textwidth]{图3_1}
\caption{自然图像及其对数幅度谱曲线~\cite{HouXiaodiCVPR2007Residual}}
\label{图3_1}
\end{figure}
\begin{figure}[!t] %插图
\centering
\includegraphics[width=0.8\textwidth]{图3_2}
\caption{不同图像的均值对数幅度谱曲线~\cite{HouXiaodiCVPR2007Residual},自左至右分别是1幅、10幅、100幅。}
\label{图3_2}
\end{figure}

视觉系统会将图像的冗余信息去除,因此从自然图像的幅度谱中减去冗余信息的幅度谱,剩下的即为显著部分的幅度谱,这就是谱剩余的方法,所以对于自然图像,能够引起感兴趣产生视觉注意的是对数幅度谱中异常的信息,用$\mathcal{R}(f)$来表示,则:
\begin{equation}
H(\mathcal{R}(f))=H(\mathcal{L}(f)|\mathcal{A}(f))
\label{式3_3}
\end{equation}
其中,$\mathcal{L}(f)$表示自然图像的对数幅度谱,先验信息$\mathcal{A}(f)$表示一般对数幅度谱,图~\ref{图3_2}已经说明一般对数幅度谱拥有局部线性的特征,因此,可以用一个$n \times n$局部均值滤波器${h}_{n}(f)$对对数幅度谱进行滤波得到一般对数幅度谱$\mathcal{A}(f)$:
\begin{equation}
\mathcal{A}(f)={h}_{n}(f)\ast \mathcal{L}(f)
\label{式3_4}
\end{equation}
因此,剩余谱$\mathcal{R}(f)$可以表示为:
\begin{equation}
\mathcal{R}(f)=\mathcal{L}(f)-\mathcal{A}(f)
\label{式3_5}
\end{equation}
最后,将取得的剩余谱与初始图像的相位谱结合进行傅里叶反变换获得显著图。该算法可以归纳如下:
\begin{linenomath}
\begin{align}
\mathcal{A}(f) &= \Re(\mathfrak{F}[\mathcal{I}(x)])\label{式3_6}\\
\mathcal{P}(f) &= \Im(\mathfrak{F}[\mathcal{I}(x)])\label{式3_7}\\
\mathcal{L}(f) &= log(\mathcal{A}(f))\label{式3_8}\\
\mathcal{R}(f) &= \mathcal{L}(f)-h_{n}(f)\ast \mathcal{L}(f) \label{式3_9}\\
\mathcal{S}(x) &= g(x)\ast \mathfrak{F}^{-1}[exp(\mathcal{R}(f)+\mathcal{P}(f))]^{2}\label{式3_10}
\end{align}
\end{linenomath}
其中,$\mathfrak{F}$和$\mathfrak{F}^{-1}$分别代表傅里叶变换和傅立叶反变换,$\mathcal{P}(f)$代表原图像的相位谱,$\Re$与$\Im$分别表示傅里叶变换得到的的实部和虚部,$g(x)$是高斯滤波器用于提高图像的显著性效果。

%=============================================================================================================
\section{相位谱四元傅里叶变换显著性检测模型(PQFT)}
\label{3_2}

尽管剩余谱算法~\cite{HouXiaodiCVPR2007Residual}既快又简单,而且能够得到较好的显著性检测结果,但是它的原因却是不明了的,而且不能确定幅度谱中的剩余谱是否真正反映了一个场景中的新颖部分或是显著区域,如图~\ref{图3_3},(a)和(c)是具有相同背景,但是图(a)有黄色头饰(显著性目标),图(c)中黄色头饰在前景中消失了,这两幅图像所对应的对数幅度谱分别为(b)和(d)。很明显,尽管这两幅图像所对应的幅度谱有微小的区别,但是很难从这两种曲线中区分图像新颖的信息(显著性物体)是什么。另外,谱剩余算法中只使用了图像的灰度信息,并没有考虑颜色等因素的影响,所以对于颜色特征区别明显而亮度变化不明显的图像该种算法效果不好。
\begin{figure}[!t] %插图
\centering
\includegraphics[width=0.8\textwidth]{图3_3.png}
\caption{场景中有无黄色头饰(显著性物体)的两幅幅度谱曲线图的对比}
\label{图3_3}
\end{figure}

基于以上分析,Guo等人在谱剩余算法的基础上提出了相位谱四元傅里叶变换算法(PQFT)~\cite{GuoChenleiCVPR2008Spatio}。在SR模型中,相位谱在计算的过程中被完整的保留了下来。因此,在PFT算法中,忽略剩余谱的处理过程,直接去掉幅度谱,只使用相位谱进行傅里叶反变换,就可以得到空间域的显著图。如图~\ref{图3_4}所示,给出一个矩形脉冲的波形,忽略幅度谱进行傅里叶反变换后,在脉冲的边缘处会产生巨大的尖峰,这是因为在脉冲的边缘是由无数的正弦函数叠加而成的; 当对于一个单一频率的正弦波后,去掉幅度谱的反变换的信号则没有明显的尖刺。
\begin{figure}[!t] %插图
\centering
\includegraphics[width=0.8\textwidth]{图3_4}
\caption{一维信号与对应信号相位谱的反变换示意图。左边是原始信号,右边是左边信号仅利用相位谱进行傅里叶反变换的重构图。}
\label{图3_4}
\end{figure}

输入一幅图像$I(x,y)$,PFT具体的处理过程如下:
\begin{linenomath}
\begin{align}
f(x,y) &= F(I(x,y))\label{式3_11}\\
p(x,y) &= P(f(x,y))\label{式3_12}\\
sM(x,y) &= g(x,y)\ast\Vert F^{-1}[e^{i\cdot p(x,y)}]\Vert^{2}\label{式3_13}
\end{align}
\end{linenomath}
其中,$F()$代表傅里叶变换,$P()$代表取傅里叶变换的相位谱,$F^{-1}[]$为傅里叶反变换,$g(x,y)$为高斯滤波器用于提高图像的显著性效果。通过实验可以证明,该方法与谱剩余方法获得的显著图几乎一样,因此也证明了剩余谱不是产生显著性的必要条件,并且PFT方法更加迅速简单。

之后,Guo等人~\cite{GuoChenleiCVPR2008Spatio}对PFT方法进行了进一步扩展,他们增加了图像的颜色特征$RG$和$BY$,并且加入了运动特征$M$,于是将图像的四个特征图用四元数表示出来,利用四元数傅里叶变换将图像转换到频域,然后将频域中的幅度谱去掉,再进行四元数傅里叶反变换转换到空间域得到最终的显著图。

%=============================================================================================================
\section{频域除法归一化显著性检测模型(FDN)}
\label{3_3}

除法归一化~\cite{BianCognNeurodyn2010Visual}可以用来模仿视觉神经细胞的侧抑制机制,它被看作是对视觉注意起关键作用的中心--周围对抗机制。在谱白化理论~\cite{Bian2009Biological}的基础上,频域除法归一化是一种基于频域、具有生物学依据,并且符合信息理论的自底向上的视觉注意显著性检测算法。该种方法首先证明了频域方法也具有生物学依据,认为视觉神经节细胞和V1区的简单细胞的特征提取,在频域中可以近似的表示为在图像频谱上用分割的带通滤波相乘。除法归一化是指对于每个具有相同种类特征的简单细胞的输出,除法归一化的输出等于每个像素输出除以其周围像素的输出加权和加上一个常数,除法归一化模型定义为:
\begin{linenomath}
\begin{align}
\hat{r}_{i}[n]=\frac{r_{i}[n]}{\sqrt{\sum_{n'}w[n']|r_{i}[n']|^{2}+\sigma^{2}}}
\label{式3_14}
\end{align}
\end{linenomath}
其中$r_{i}[n']$是$r_{i}[n]$周围的输出,$w$是权值矩阵,$\sigma$是一个常数。频域除法归一化的处理步骤如下:
\begin{enumerate}
\item 变换图像尺寸并将自然图像从RGB彩色空间转到Lab彩色空间,然后将图像分为3通道:$s_{c}|c\in\{l,a,b\}$;
\item 取每个通道$s$进行傅里叶变换:
\begin{linenomath}
\begin{align}
S[k]=\sum_{x}s(x)e^{-2\pi kx/N}
\label{式3_15}
\end{align}
\end{linenomath}
\item 利用频域分解方法把每个傅里叶系数$k$归到一个特征$i$;
\item 计算每个特征$i$的除法归一化分母$E_{i}$,该文献中将两个归一化参数$w$和$\sigma$都设置为1:
\begin{linenomath}
\begin{align}
E_{i}=\sqrt{w\sum_{k\in i}|S[k]|^{2}/N+\sigma^{2}}
\label{式3_16}
\end{align}
\end{linenomath}
\item 创建分母图$E[k]$:$E[k]=\{E_{i}\in i\}$
\item 计算频域除法归一化的结果值:
\begin{linenomath}
\begin{align}
\hat{S}[k]=\frac{S[k]}{E[k]}
\label{式3_17}
\end{align}
\end{linenomath}
\item 取每个频谱$\hat{S}$的傅里叶反变换:
\begin{linenomath}
\begin{align}
\hat{s}(x)=W(x)|\sum_{k}\hat{S}[k]e^{2\pi kx/N}|^{2}
\label{式3_18}
\end{align}
\end{linenomath}
\item 最后从所有通道$c$中求出最终的显著图$sM$:
\begin{linenomath}
\begin{align}
sM(x)=max\{\hat{s}_{c}(x)\}
\label{式3_19}
\end{align}
\end{linenomath}
\item 利用高斯滤波器对显著图进行平滑处理。
\end{enumerate}

%=============================================================================================================
\section{基于相位谱和调谐幅度谱的显著性检测模型(PTA)}
\label{3_4}

针对谱剩余算法得到的显著图对比度比较差、细节显著性检测结果不太理想的问题,李崇飞等人~\cite{李崇飞2012相位谱}经过分析图像频谱特点和显著性的关系,提出了一种基于频谱分析的显著性区域检测方法。该算法分析了图像数据进行傅里叶变换之后,显著信息和非显著信息都会以统计信息的方式保留在傅里叶变换的频谱中,通过证明相位谱对于图像显著信息的重要意义,该方法保留了相位谱,然后该文献在分析了幅度谱中冗余信息和显著信息的统计特性后,提出了幅度谱分段调谐的方法,即抑制幅度谱的高频分量,增强幅度谱的低频分量,达到抑制图像大部分冗余信息并且保留并增强图像的显著性信息的效果。

PTA方法~\cite{李崇飞2012相位谱}认为幅度谱是各种基波的加权和,非显著性信息的幅值(比重比较大)在幅度谱中表现为比较高的幅值,需要对它们进行抑制; 而代表显著信息的幅值(比重比较小)在幅度谱中表现为比较低的幅值,需要对它们进行保留和增强。PTA~\cite{李崇飞2012相位谱}通过利用阈值的方法确定了显著性特征和非显著性特征在频域幅度谱的分界(边界),自适应抑制了大于阈值的幅度谱,自适应增强了小于阈值的幅度谱,从而突出了显著性特征。对于幅度谱$a(u,v)$显著性分类阈值$\Delta$为:
\begin{linenomath}
\begin{align}
\Delta=\gamma M(a(u,v))
\label{式3_20}
\end{align}
\end{linenomath}
幅度谱调谐$a'(u,v)$为:
\begin{linenomath}
\begin{align}
a'(u,v) = \left\{ \begin{array}{ll}
f(u,v)/(|a(u,v)-\Delta|^{\alpha}) & a(u,v)>\Delta\\
f(u,v)|a(u,v)-\Delta|^{\beta}) & a(u,v)\leq \Delta
\end{array} \right.
\label{式3_21}
\end{align}
\end{linenomath}
其中,$f(u,v)$为图像傅里叶变换后的值,$\alpha$和$\beta$为参数,依据图像的对比度通过实验进行选择,一般如果图像的对比度比较大时,即像素方差较大,则$\alpha$取值比较大而$\beta$的值较小,反之则$\alpha$取值比较小而$\beta$的值较大。给定一幅图像$I(x,y)$,算法的结构如下:
\begin{linenomath}
\begin{align}
f(u,v) &= F(I(x,y))\label{式3_22}\\
p(u,v) &= P(f(u,v))\label{式3_23}\\
a(u,v) &= \Vert f(u,v)\Vert \label{24}\\
I'(x,y) &= \Vert F^{-1}[exp(log(a'(u,v)+i\cdot p(u,v)))]\Vert^{2}\label{式3_25}\\
S(x,y) &= G\ast I'(x,y)\label{式3_26}
\end{align}
\end{linenomath}

基于相位谱和调谐幅度谱的显著性检测方法在显著图对比度和显著性细节检测等方面比谱剩余方法的性能效果有了更大的提升。

%=============================================================================================================
\section{基于图像签名的显著性检测模型(SIG)}
\label{3_5}
图像签名~\cite{HouXiaodiTPAMI2012Signature}是由侯晓迪等人最早提出的,它在整个频域范围内丢弃了信号的幅度谱信息,仅仅保留了图像的离散余弦变换部分的符号,即保留了图像的相位谱。在~\cite{Hou2014Phd}中已经被证明了图像重要的视觉信息主要保留在幅度谱中。该描述符是基于信号的稀疏性理论提出的,近年来稀疏性理论已被广泛的应用在数字信号和计算机视觉等领域,图像稀疏性理论认为,图像都是稀疏的。信号经过稀疏变换后,绝大多数有用的信息都将集中分布在少量的变换系数中~\cite{郑源彩2012稀疏},常见的图像稀疏变换的方法有小波变换、离散余弦变换和有限差分等,图像签名即采用的离散余弦变换。图像签名的定义为:
\begin{linenomath}
\begin{align}
\textrm{ImageSignature$(x)$=sign$DCT(x)$}
\label{式3_27}
\end{align}
\end{linenomath}

侯晓迪~\cite{HouXiaodiTPAMI2012Signature}证明了图像签名可以用来粗略的获得图像前景的位置,并且提出了两个命题:
\begin{enumerate}
\item 图像签名的重构图像可以用来从稀疏背景中估计图像稀疏性的前景位置,即图像签名可以抑制图像的背景;
\item 对于一个非零元素的前景信号$f$,通过图像签名重构后的信号有$79\%$的信息保留在前景信号中。
\end{enumerate}
其证明过程如图~\ref{图3_5}所示,给定一幅图像$f$,它可以分为前景信号$f$和背景信号$b$,即
\begin{linenomath}
\begin{align}
x=f+b&&x,f,b\in\mathbb{R}^{N}
\label{式3_28}
\end{align}
\end{linenomath}
$f$经过离散余弦变换可以得到变换后的前景$\hat{f}$,背景$b$经过离散余弦变换可以得到变换后的背景$\hat{b}$,通过相加可以得到离散余弦变换后的信号$\hat{x}$,对$\hat{x}$取符号运算即可得到信号的图像签名。根据上述命题1的理论,此时信号$\hat{x}$的符号值与离散余弦变换的前景$\hat{f}$的符号值是相近的,因为前景信号$f$和离散余弦变换背景信号$\hat{b}$都只有很少数量的非零分量。将$sign(\hat{f})$进行离散余弦反变换得到重构信号$\bar{f}$,然后根据上述命题2可得,重构后的信号$\bar{f}$与前景信号$f$是近似的,进而有$\bar{x}$近似等同于$f$,从而证明了图像签名可以用来粗略的获得图像前景的位置,并可以近似估计前景的大小。
\begin{figure}[h] %插图
\centering
\includegraphics[width=0.8\textwidth]{图3_5}
\caption{图像签名的证明框架~\cite{HouXiaodiTPAMI2012Signature}}
\label{图3_5}
\end{figure}

利用图像签名算子重构后的图像可以检测空间中嵌入在稀疏背景中的前景信号,重构后的信号与人类感兴趣(产生视觉注意)的区域是重叠的,因此可以用来进行图像的显著性检测。给定图像$x$,rgb-SIG显著性检测算法如下:
\begin{enumerate}
\item 输入图像$x$,将图像的大小转换为一定的大小,并提取图像的R、G和B颜色特征$x^{i} (i=1,2,3)$;
\item 取每个通道$x^{i}$进行离散余弦变换,再求$sign$值:
\begin{linenomath}
\begin{align}
\textrm{ImageSignature$(x^{i})$=sign$(DCT(x^{i}))$}
\label{式3_29}
\end{align}
\end{linenomath}
\item 对变换后的图像进行离散余弦反变换重构:
\begin{linenomath}
\begin{align}
\bar{x^{i}}=IDCT[sign((DCT(x^{i})))]
\label{式3_30}
\end{align}
\end{linenomath}
\item 利用高斯滤波对显著图进行处理:
\begin{linenomath}
\begin{align}
m=g\ast\sum_{i}(\bar{x}^{i}\circ\bar{x}^{i})
\label{式3_31}
\end{align}
\end{linenomath}
\item 最后对$m$取均值获得最终的显著图。
\end{enumerate}

值得注意的是,图像签名是计算机视觉领域中的一个简单有力的算子,可以用来估计图像前景的位置和尺寸的大小,显著性检测是该算子的一个具体应用,在后面本文的算法中也会用到图像签名算子。

%=============================================================================================================
\section{基于频域尺度空间分析的显著性检测模型(HFT)}
\label{3_6}

显著性目标往往是独特和不规则的,这些特征很容易与周围背景区分开来,很多算法是通过检测图像的显著性计算的。基于频域尺度空间分析的显著性检测方法~\cite{LiJianTPAMI2013Scale}并没有试图寻找独特不规则的显著性目标,而是通过分析非显著性区域的规则特性在频域中的表现,进而抑制非显著性区域来进行显著性检测。

该模型假设自然图像是由显著性区域和所谓的规则的非显著性区域构成的,而非显著性区域表现为重复性模式,通过傅里叶变换的幅度谱可知,规则的非显著性区域在幅度谱上表现为尖峰,并且重复的模式越多尖峰越明显。因此,为了抑制非显著性区域,可以通过抑制幅度谱中的尖峰进行处理。该模型通过利用高斯滤波器在合适尺度上对幅度谱进行平滑操作,从而在空间域中抑制了重复模式。不同的滤波尺度对检测显著性的要求不一样,小的滤波尺度可以用来检测大的显著性区域,大的滤波尺度可以用来检测小的显著性区域。

为了解决尺度选择的问题,HFT模型提出了谱尺度空间(SSS)~\cite{LiJianTPAMI2013Scale},即通过一系列的高斯滤波簇对图像的幅度谱进行滤波,得到一系列滤波后的幅度谱。给定一幅图像的幅度谱$A(u,v)$,谱尺度空间是将$A(u,v)$与一系列的高斯核卷积获得的衍生信号$\Lambda$。选择的高斯滤波簇为:
\begin{linenomath}
\begin{align}
g(u,v;k)=\frac{1}{\sqrt{2\pi}2^{k-1}t_{0}}e^{(u^2+v^2)/(2^{2k-1}t_{0}^{2})}
\label{式3_32}
\end{align}
\end{linenomath}
其中$k$是尺度参数,$k=1,\dots,K$,$K$与图像的尺寸有关:$K=[log_{2}min\{H,W\}]+1$,$H$和$W$分别是图像的高度和宽度。$t_{0}$为一个参数。因此,尺度空间定义为:
\begin{linenomath}
\begin{align}
\Lambda(u,v;k)=(g(.,.;k)\ast A)(u,v)
\label{式3_33}
\end{align}
\end{linenomath}

通过计算得到图像的尺度空间,然后与图像的相位谱(相位谱不变)进行傅里叶反变换进而获得一系列显著图,如何对这些生成的显著图进行选择呢?HFT模型提出了运用图像熵的方法选出效果较好的显著图作为最终的显著图。图像熵~\cite{Abutaleb1989Automatic}用来统计图像的特征,主要反映了图像平均信息量的多少,表征图像灰度分布的聚集特性。灰度图像的一元灰度熵为:
\begin{linenomath}
\begin{align}
H=\sum_{i=0}{255}p_{i}log p_{i}
\label{式3_34}
\end{align}
\end{linenomath}
其中$p_{i}$是某个灰度值在图像中出现的概率,该值可以通过图像直方图取得。尽管图像一维熵可以用来反映灰度分布的聚集特性,但是不能够反映灰度空间分布特性。为了表示该特征,HFT模型引入了图像二维熵,它是通过选择图像的邻域灰度值来作为灰度值分布的空间几何特征值,该模型进行了简单的处理,将图像的一维熵分布变量值$x$与高斯滤波器$g_{n}$进行卷积得到新的二维熵$H_{2D}$:$H_{2D}(x)=H\{g_{n}\star x\}$。二维熵可以表示图像中像素的位置灰度信息与邻域内灰度值分布信息的综合特性。HFT模型认为,二维熵值越小,图像的显著性效果越好,该模型选取了熵值最小的显著图作为最终的显著图。给定尺寸为$m×n$的图像$C$,基于频域尺度空间分析的显著性检测算法结构如下:
\begin{enumerate}
\item 提取图像的特征图$\{I,RG,BY\}$;
\item 根据图像的特征图,计算图像的四元傅里叶变换,获得图像的幅度谱$A$,相位谱$P$和特征轴谱$\chi$;
\item 利用高斯滤波簇对图像的幅度谱进行平滑滤波得到图像的尺度空间$\{\Lambda\}$;
\item 通过四元数傅里叶反变换将尺度空间中的衍生幅度谱与图像的相位谱反变换到空间域,得到一系列显著图;
\item 利用图像的二维熵选择合适的显著图作为最终的显著图$S$。
\end{enumerate}

%=============================================================================================================
\section{本章小结}
\label{3_7}

本章我们介绍了国内外几种典型的频域显著性检测算法,包括SR模型、PQFT模型、FDN模型、PTA模型、SIG模型和HFT模型,分别对他们提出动机、算法描述和表示意义上进行了详细的解释和说明。
