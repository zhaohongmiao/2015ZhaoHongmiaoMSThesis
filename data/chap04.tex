%%% Local Variables: 
%%% mode: latex
%%% TeX-master: t
%%% End: 

\chapter{基于幅度谱分析的视觉显著目标检测}
\label{cha4}

通过上一章的分析我们知道,谱剩余算法~\cite{HouXiaodiCVPR2007Residual}和相位谱傅里叶变换算法~\cite{GuoChenleiCVPR2008Spatio}在一定程度上等价于一个梯度算子与高斯后续处理的结合。这是因为自然图像的幅度谱在低频区域的幅度值远远高于在高频区域的幅度值。如图~\ref{图4_1}所示,原图像的幅度谱绝大部分集中在低频区域,经过谱剩余处理后,(b)的幅度谱值比较均匀,高频区域的幅度谱和低频区域的幅度谱相差不大,这就类似等同于高频区域和低频区域分布量几乎相等,而傅里叶变换相位谱算法的处理则是将整个幅度谱设为同一个值,即将幅度谱处理为一个平面,这种处理换句话说就是抑制了幅度谱的低频区域,增强了幅度谱的高频区域。基于以上的分析可知,通过傅里叶反变换可以看出谱剩余算法和相位谱傅里叶变换算法获得的显著图几乎一样,检测的都是图像的边缘和纹理密集的区域,这也说明了他们的算法仅仅可以用来检测中央周围对比度比较强烈的小的显著区域,对于大的显著区域的检测则比较困难。
\begin{figure}[h]
  \centering
  \includegraphics[height=9.8cm]{图4_1}
  \caption{SR算法和PQFT算法与原图像频域及显著图的对比图。(a)是元图像的变换; (b)是SR模型的变换; (c)是PQFT模型的变换。第一列:原图像; 第二列:频谱图; 第三列:对频谱图取log后的幅度谱图; 第四列:反变换后的图像。}
  \label{图4_1}    
\end{figure}

对于一个比较好的显著目标检测方法,其应该具备以下特点~\cite{AchantaCVPR2009Frequency}:
\begin{enumerate}
\item 小的显著目标和大的显著目标都可以检测;
\item 可以均匀的突出整个显著目标,而不仅仅突出目标的边缘或纹理较密集的区域;
\item 可以将显著目标从复杂的背景中检测出来;
\item 检测效率高,具有高效的图像处理能力,适应工程性应用。
\end{enumerate}

谱剩余算法和傅里叶变换相位谱算法以及在上一章节中介绍的其他算法不能同时满足上述所有的特点,如谱剩余算法~\cite{HouXiaodiCVPR2007Residual}和傅里叶变换相位谱算法~\cite{GuoChenleiCVPR2008Spatio}满足第3项和第4项特点,不满足第1、2项; PTA方法~\cite{李崇飞2012相位谱}满足1、3、4项,不满足第2项等。我们受到了SSS方法~\cite{LiJianTPAMI2013Scale}的启发,经过仔细的分析和实验,提出了一种新的频域显著性处理方法,通过与其他频域算法的对比,我们的方法可以获得更好的显著目标检测效果。

%=============================================================================================================
\section{幅度谱分析}
\label{4_1}

根据第~\ref{cha2}章的分析可知,图像通过傅里叶变换后,信息都将保留在变换后的幅度谱和相位谱中,幅度谱代表了图像的明暗对比度信息,而图像的结构信息包含在相位谱中,因此图像的相位谱在变换中是不能进行改变的,要保留下来,图像频域显著性检测主要是对信号的幅度谱进行处理。

%-------------------------------------------------------------------------------------------------------------
\subsection{图像的重复模式分析}
\label{4_1_1}

根据Li等人~\cite{LiJianTPAMI2013Scale}提出的观点,给定一幅图像,它可以被看作是由不同的小区域组成,而每个小区域可以称为一种模式。图~\ref{图4_2}是中国海洋大学作宣传片时用的图片(美丽的海洋),从所给的图像中我们可以看出图像被分成了不同的小区域,很多分块的小区域是非常相似的,或者说是重复区域,例如分块的海洋小区域,而海洋上的小岛分块之后,他们的小区域是不重复或独特的。根据视觉注意机制,反复出现的重复模式一般不会引起人类产生视觉注意,而异常或不重复出现的模式常常引起注意,视觉系统往往会将注意力集中在少数感兴趣的区域(小岛),将图像中的冗余信息(海洋)去除。所以我们的目的是通过分析图像,找到重复模式的特征,抑制重复模式,从而突出显著区域。频域处理可以有效的实现这个目地。
\begin{figure}[h]
  \centering%
  \begin{subfigure}{0.38\textwidth}
    \includegraphics[height=3.9cm]{图4_2_a}
    \caption{}
  \end{subfigure}
  \hspace{4em}%
  \begin{subfigure}{0.45\textwidth}
    \includegraphics[height=5cm]{图4_2_b}
    \caption{}
  \end{subfigure}
  \caption{图像的重复模式}
  \label{图4_2}
\end{figure}

假设信号用$f(t)=\sum_{n=-\propto}^{\propto}F(n)e^{jnw_{1}t}$,其中$F(n)=\frac{1}{T}\int_{-T/2}^{T/2}f(t)e^{-jnw_{1}t}$,其傅里叶表达式为:
\begin{linenomath}
\begin{align}
\mathcal{F}(w)=2\pi\sum_{n=-\propto}^{\propto}F(n)\delta(w-nw_{1})
\label{式4_1}
\end{align}
\end{linenomath}
根据傅里叶变换可知,空间域或时域中周期出现的信号,对应于频域幅度谱中一个或者多个频率位置的尖峰。例如图~\ref{图4_3}中一维信号所示,图中左侧分别给出了3种不同的有限长信号,信号中重复模式(周期)的长度是不同的,图像的右侧是左侧不同信号所对应的幅度谱。从图中可以看出随着重复模式的增加,频域幅度谱中尖刺越来越尖。
\begin{figure}[h]
  \centering
  \includegraphics[height=8cm]{图4_3}
  \caption{一维信号不同重复模式所对应的幅度谱的尖峰。左边是原始信号,右面是对应信号的幅度谱。}   
  \label{图4_3} 
\end{figure}
同理,二维信号同样具有该种特点,图~\ref{图4_4}是由不同重复模式的二维信号所对应的3种不同的图像,图像中重复模式的大小是不同的,我们同样可以看出随着重复模式的增加,频域对数幅度谱(为了处理及计算方便,二维图像我们取的是对数幅度谱)中的尖刺越来越尖。如果用高斯低通滤波器对不同重复模式信号的幅度谱进行平滑滤波,尖峰越大的下降的数值越大~\cite{LiJianTPAMI2013Scale}。我们可以利用信号的上述特点对图像的幅度谱进行高斯平滑处理,滤掉代表重复模式的非显著区域,通过进一步分析和实验进行显著区域检测。
\begin{figure}[h]
  \centering
  \includegraphics[height=9cm]{图4_4}
  \caption{二维信号不同重复模式所对应的幅度谱的尖峰。左边表示原始图像,右面表示图像所对应的对数幅度谱。}   
  \label{图4_4} 
\end{figure}

%-------------------------------------------------------------------------------------------------------------
\subsection{幅度谱滤波抑制重复模式分析}
\label{4_1_2}

Li等人~\cite{LiJianTPAMI2013Scale}提出了通过高斯核$h$可以来抑制图像幅度谱$\Vert \mathcal{F}\{f\}\Vert$中的尖刺:
\begin{linenomath}
\begin{align}
\mathcal{A}_{s}(u,v)=\Vert \mathcal{F}\{f(x,y)\}\Vert\ast h
\label{式4_2}
\end{align}
\end{linenomath}
滤波后的幅度谱$\mathcal{A}_{s}$和初始的相位谱$\mathcal{P}$结合计算其傅里叶反变换,得到图像的显著图$S$:
\begin{linenomath}
\begin{align}
S=\mathcal{F}^{-1}\{\mathcal{A}_{s}(u,v)e^{i\mathcal{P}(u,v)}\}
\label{式4_3}
\end{align}
\end{linenomath}
为了提高视觉显著效果进行了后期处理,得到图像的最终显著图:
\begin{linenomath}
\begin{align}
S=g*|\mathcal{F}^{-1}\{\mathcal{A}_{s}(u,v)e^{i\mathcal{P}(u,v)}\}|^{2}
\label{式4_4}
\end{align}
\end{linenomath}
该处理过程可以简单用一维信号来模拟,如图~\ref{图4_5},输入的信号(第一行)是一个周期信号,但是在该段信号中有一小部分是频率与背景信号不同、从背景中突显出来的显著信号,该小部分显著信号会使人产生视觉注意,显著目标检测的目的就是要将这一段突出的信号检测出来并对其进行均匀的突出显示。第二行是信号经过傅里叶变换得到的幅度谱取对数获得的信号,我们可以看到信号中有三个较为明显的尖峰(图中用三种颜色的框框出),红色框中的尖峰对应于图像的直流分量,绿色框中的尖峰对应于重复模式较多的背景信号,我们可以看到这两种信号的尖刺都比较尖且高,蓝色框中的尖峰对应的是重复模式较少的显著信号,其尖峰值相对较小。利用高斯核卷积对信号的对数幅度谱进行平滑滤波得到第三列表示的新的幅度谱,然后将滤波后的幅度谱与信号的相位谱进行傅里叶反变换得到重构后的信号(第四行(d))。我们可以明显的看到信号的背景受到了极大的抑制,从而突出了显著信号。最后对显著信号进行后续增强处理得到增强后的显著信号(第五行(e))。第六行(f)代表被平滑掉的幅度谱,即重复模式的背景信号所对应的幅度谱。最后一行(g)代表被移除后的幅度谱(e)与图像原始相位谱结合经过傅里叶反变换得到的信号。经过上面的分析我们可以看出,利用幅度谱滤波可以很好的抑制信号中的非显著性或背景区域,这个处理也说明了频域幅度谱与一定尺度的高斯核进行卷积等同于图像的显著性检测。
\begin{figure}[h]
  \centering
  \includegraphics[height=8.5cm]{图4_5}
  \caption{通过幅度谱滤波抑制重复模式示意图}   
  \label{图4_5} 
\end{figure}

%--------------------------------------------------------------------------------------------------------------
\subsection{最优尺度选择分析}
\label{4_1_3}

通过选择合适的高斯滤波尺度平滑图像的幅度谱可以用来抑制非显著区域,但是如何来选择正确的尺度呢?Li等人~\cite{LiJianTPAMI2013Scale}已经证明了滤波器尺度的选择对最后重构获得的信号的影响。如图~\ref{图4_6},第一行是原始信号与其所对应的频域幅度谱,第二、三、四和五行分别是通过不同尺度的高斯滤波器对信号的幅度谱进行滤波得到的显著性检测结果以及他们所对应的滤波后的幅度谱,并且从上到下滤波尺度越来越大。从图中可以看出,如果选择的滤波尺度太小,则重复性模式(背景)得不到有效的抑制(第二行); 如果选择的滤波尺度太大,则仅有显著性的边缘或纹理密集的区域被突出出来,显著目标突显的不均匀(第四行和第五行)。因此,选择一个合适尺度的高斯滤波器对于显著区域检测具有十分重要的影响,同时可以得出:小的滤波尺度核可以用来检测大的显著区域,大的滤波尺度核可以用来检测小的或纹理密集的显著区域。
\begin{figure}[h]
  \centering
  \includegraphics[height=6.5cm]{图4_6}
  \caption{滤波尺度对显著性检测的影响}   
  \label{图4_6} 
\end{figure}

为了解决上述的问题,HFT模型~\cite{LiJianTPAMI2013Scale}提出了尺度空间的概念,即通过一系列不同尺度的高斯滤波簇对图像的幅度谱进行滤波,得到一系列滤波后的幅度谱,具体可以参考第~\ref{3_6}节中的内容。获得谱尺度空间后,Li再将滤波后的幅度谱簇与原始图像的相位谱进行傅里叶反变换得到对应的一系列显著图,最后通过计算显著图的二维熵得到最佳尺度。通过实验我们发现,利用二维熵选择最佳尺度的方法并不好,如图~\ref{图4_7}所示,假设一幅图像中整个圆形区域是显著区域,(a)和(b)分别是不同尺度下对应的显著图,一个突出了整个圆形区域,另一个是突出了边缘的圆环。通过计算二维熵可得,圆形的熵为3.25,圆环的熵为2.99,根据SSS模型最优选取标准,应当选取熵值小的图像作为最终显著图。因此,他们会将圆环图像作为最终显著图,很显然这种选择是不正确的。另外,对于一幅图像中存在多个目标情况,由于SSS模型只从多个尺度对应的显著图中选择一个,所以将丢失部分显著信息,使得最终显著图不能将多个显著区域均匀的突出出来。本文针对这些问题改进了新的算法,由检测结果~\ref{图4_8}可以看出,相比HFT模型,我们的算法能得到更均匀更精确的显著区域。
\begin{figure}[t]
  \centering%
  \begin{subfigure}{0.39\textwidth}
    \includegraphics[height=4.5cm]{图4_7_a}
    \caption{}
  \end{subfigure}
  \hspace{4em}%
  \begin{subfigure}{0.4\textwidth}
    \includegraphics[height=4.5cm]{图4_7_b}
    \caption{}
  \end{subfigure}
  \caption{突出整个目标和突出目标轮廓的显著图}
  \label{图4_7}
\end{figure}
\begin{figure}[b]
  \centering%
  \begin{subfigure}{4cm}
    \includegraphics[height=3cm]{图4_8_a.eps}
    \caption{}
  \end{subfigure}
  \hspace{4em}%
  \begin{subfigure}{0.25\textwidth}
    \includegraphics[height=3cm]{图4_8_b.eps}
    \caption{}
  \end{subfigure}
  \hspace{4em}%
  \begin{subfigure}{0.25\textwidth}
    \includegraphics[height=3cm]{图4_8_c.eps}
    \caption{}
  \end{subfigure}
  \hspace{4em}%
  \begin{subfigure}{0.25\textwidth}
    \includegraphics[height=3cm]{图4_8_d.eps}
    \caption{}
  \end{subfigure}
  \caption{HFT模型检测结果与我们的方法检测结果对比图}
  \label{图4_8}
\end{figure}

Li等人~\cite{LiJianTPAMI2013Scale}已经证明了如果滤波尺度($\sigma$)太小,重复模式的背景区域就不能得到有效的抑制;如果滤波尺度($\sigma$)太大,则只能突出图像显著目标的边缘或纹理密集的区域。因此,选择一个合适尺度的高斯核滤波器非常重要。为了找到正确的尺度,我们对前景背景尺寸比值与最优尺度之间的关系进行了研究。

为了便于理解,我们利用一维信号作为例子。图~\ref{图4_9}的第一行是前景(显著目标)长度为$l$,背景重复模式长度为$L$的信号,然后将该信号进行傅里叶变换,得到频域的幅度谱,通过一系列不同尺度的高斯核$g(x;\sigma)$与幅度谱进行卷积获得频域的幅度谱空间,高斯核为:
\begin{linenomath}
\begin{align}
g(x;\sigma)=\frac{1}{\sqrt{2\pi}\sigma}e^{\frac{x^{2}}{2\sigma^{2}}}
\label{式4_5}
\end{align}
\end{linenomath}
其中,$\sigma$是标准差,代表高斯核的滤波尺度。图~\ref{图4_9}中其余的行代表尺度空间中不同尺度高斯滤波器滤波获得的幅度谱与原始信号的相位谱进行傅里叶反变换而得到的一系列重构显著信号,且重构的显著信号越接近前景信号,显著性就越好。我们定义了相似度$\gamma$用来量化前景信号与重构信号的相似性:
\begin{linenomath}
\begin{align}
\gamma=\frac{\sum_{t=1}^{l}(S_{t}-S'_{t})^{2}}{\sum_{t=1}^{l}S_{t}^{2}}
\label{式4_6}
\end{align}
\end{linenomath}
其中,$S_{t}$表示前景信号,$S'_{t}$表示重构的显著信号,$\gamma$最小的值对应的重构信号(图~\ref{图4_9})与图像的前景最相似,说明该信号所对应的滤波尺度为最优尺度。因此,我们画出了前景和背景长度比值$l/L$与最优滤波尺度$\sigma$之间的关系,如图~\ref{图4_10}中的蓝色曲线。从该曲线中我们可以容易地发现小尺度核的高斯滤波器可以检测大的显著目标,大尺度核的高斯滤波器可以检测小的显著目标。通过曲线拟合(见图~\ref{图4_10}的红色曲线),得到$l/L$和$\sigma$的特定关系:
\begin{linenomath}
\begin{align}
\sigma=\alpha\cdot(l/L)^{(-1)}
\label{式4_7}
\end{align}
\end{linenomath}
其中$\alpha$是一个调节系数。
\begin{figure}[h]
  \centering
  \includegraphics[height=9.8cm]{图4_9.pdf}
  \caption{最优滤波尺度对应的重构信号的选择}   
  \label{图4_9} 
\end{figure}
\begin{figure}[h]
  \centering
  \includegraphics[height=7cm]{图4_10.eps}
  \caption{图像前景和背景的尺寸比率$l/L$与最优尺度$\sigma$关系曲线(蓝色曲线)及它的拟合曲线(红色曲线)}   
  \label{图4_10} 
\end{figure}

同理,该种规则同样适用于二维信号。幅度谱滤波的二维高斯核为:
\begin{align}
g(u,v;\sigma) &= \frac{1}{\sqrt{2\pi}\sigma}e^{\frac{(u^{2}+v^{2})}{2\sigma^{2}}}\label{式4_8}\\
\sigma &= \alpha\cdot\Big(\frac{f(h,w)}{f(H,W)}\Big)^{-1}\label{式4_9}
\end{align}
其中,$\sigma$是高斯核的滤波尺度,$H$和$W$分别是图像的高度和宽度,$h$和$w$分别代表显著目标的高度和宽度,$f$是高度和宽度的对应函数,$\alpha$是一个调节函数。

在显著目标检测中,幅度谱尺度空间的最优尺度分析为获得幅度谱最优滤波尺度提供了量化的策略。

%=============================================================================================================
\section{基于幅度谱分析频域显著目标检测算法}
\label{4_2}

我们的方法首先结合图像的颜色和亮度信息,创建四元傅里叶变换,通过Image Signature算子估测每个显著区域的大小,然后根据最优滤波尺度与显著区域尺寸大小之间的关系(公式~\ref{式4_9})自动获得对应的最优滤波尺度,最后由中央偏见策略~\cite{JuddMIT2012Benchmark}将傅里叶反变换后的显著图融合成最终显著图。对于具有$N$个显著目标的图像,本文的算法框架见图~\ref{图4_11}。
\begin{figure}[h]
  \centering
  \includegraphics[height=9.8cm]{图4_11}
  \caption{基于幅度谱分析的显著目标检测框架}   
  \label{图4_11} 
\end{figure}

%--------------------------------------------------------------------------------------------------------------
\subsection{图像四元数傅里叶变换}
\label{4_2_1}

四元数是普通复数表达式的一种推广形式~\cite{Hamilton1866book},近年来,四元数傅里叶变换越来越多的被应用到彩色图像中~\cite{EllPhD1992Hypercomplex}。有关四元数及其变换形式可以参考第~\ref{2_2_4}节中的内容,为了表达方便,下面通过简单回顾四元数的基本概念对其在彩色图像中的应用进行具体介绍。

四元数$q$是由一个实数部分和三个虚数部分构成的,其表达式为:
\begin{linenomath}
\begin{align}
q=a+b\mu_{1}+c\mu_{2}+d\mu_{3}
\label{式4_10}
\end{align}
\end{linenomath}
其中,$\mu_{1}$、$\mu_{2}$和$\mu_{3}$代表三个虚数,$\mu_{1}\mu_{2}=\mu_{3}$,$\mu_{2}\mu_{3}=\mu_{1}$,$\mu_{3}\mu_{1}=\mu_{2}$;$\mu_{1}\perp\mu_{2}$,$\mu_{2}\perp\mu_{3}$,$\mu_{3}\perp\mu_{1}$。$a$、$b$、$c$和$d$是四个实数。对于时间为$t$,位置为$(x,y)$的一帧彩色图像$I(x,y;t)$,四元数的四个实数$a$、$b$、$c$、$d$可以被图像包含的四种特征代替,它们分别是:$M(x,y;t)$为运动特征,存在于人类视皮层中的两对拮抗通道:通道$RG(x,y;t)=R(x,y;t)-G(x,y;t)$、通道$BY(x,y;t)=B(x,y;t)-Y(x,y;t)$,通道$I(x,y;t)$为图像的亮度信息。令$f_{1}(x,y;t)=M(x,y;t)+RG(x,y;t)\mu_{1}$,$f_{2}(x,y;t)=BY(x,y;t)+I(x,y;t)\mu_{1}$,我们可以将这四个特征结合起来用四元数表示成辛几何的形式:
\begin{linenomath}
\begin{align}
q(x,y;t) &= M(x,y;t)+RG(x,y;t)\mu_1+BY(x,y;t)\mu_2+I(x,y;t)\mu_3\nonumber\\
 &= f_1(x,y;t)+f_2(x,y;t)\mu_2
\label{式4_11}
\end{align}
\end{linenomath}
$r(x,y)$,$g(x,y)$和$b(x,y)$分别代表图像的红、绿、蓝三种颜色通道。根据颜色的特征表示,则有:
\begin{linenomath}
\begin{align}
R(x,y;t) &= r(x,y;t)-\frac{g(x,y;t)+b(x,y;t)}{2}\label{式4_12}\\
G(x,y;t) &= g(x,y;t)-\frac{r(x,y;t)+b(x,y;t)}{2}\label{式4_13}\\
B(x,y;t) &= b(x,y;t)-\frac{r(x,y;t)+g(x,y;t)}{2}\label{式4_14}\\
Y(x,y;t) &= \frac{r(x,y;t)+g(x,y;t)}{2}-\frac{\left|r(x,y;t)-g(x,y;t)\right|}{2}-b(x,y;t)\label{式4_15}\\
I(x,y;t) &= \frac{r(x,y;t)+g(x,y;t)+b(x,y;t)}{3}\label{式4_16}
\end{align}
\end{linenomath}
鉴于本文研究的是静态图像而非视频,所以运动通道$M(x,y;t)$和时间$t$都设置为0。因此,图像$q(x,y)$的四元傅里叶变换可以写为:
\begin{linenomath}
\begin{align}
Q[u,v] &= F_1[u,v]+F_2[u,v]\mu_2\label{式4_17}\\
F_i[u,v] &= \frac{1}{\sqrt{MN}}\sum_{x=0}^{M-1}\sum_{y=0}^{N-1}e^{-j 2\pi\left(\frac{ux}{M}+\frac{vy}{N}\right)}f_{i}(x,y)\quad i=1,2
\label{式4_18}
\end{align}
\end{linenomath}
其中,$(x,y)$和$(u,v)$分别是图像大小为$M×N$的空间域和时域的位置。为了重构图像,$Q[u,v]$表示为极坐标的形式为:
\begin{linenomath}
\begin{equation}
  \label{式4_19}
  Q[u,v]=A(u,v)e^{\mu(u,v) P(u,v)}
\end{equation}
\end{linenomath}
其中,$A(u,v)$代表图像的幅度谱,$P(u,v)$代表图像的相位谱,$\mu(u,v)$代表图像的本征轴谱,他们分别定义为:
\begin{linenomath}
\begin{align}
A(u,v) &= \left\Vert Q[u,v] \right\Vert \label{式4_20}\\
P(u,v) &= \tan^{-1}\frac{\left\Vert I\left(Q[u,v]\right)\right\Vert}{R\left(Q[u,v]\right)} \label{式4_21}\\
\mu(u,v) &= \frac{I\left(Q[u,v]\right)}{\left\Vert I\left(Q[u,v]\right)\right\Vert} \label{式4_22}
\end{align}
\end{linenomath}
其中$R\left(Q[u,v]\right)$和$I\left(Q[u,v]\right)$分别是$Q[u,v]$的实部和虚部,$\Vert\cdot\Vert$代表四元数每个元素的模值,则四元傅里叶反变换为:
\begin{linenomath}
\begin{equation}
  \label{式4_23}
  S=Q^{-1}\left\{A(u,v)e^{\mu(u,v)P(u,v)}\right\}
\end{equation}
\end{linenomath}
其中,$Q^{-1}$是傅里叶反变换可用如下式子表示:
\begin{linenomath}
\begin{equation}
f_i(x,y) = \frac{1}{\sqrt{MN}}\sum_{u=0}^{M-1}\sum_{v=0}^{N-1}e^{j 2\pi\left(\frac{ux}{M}+\frac{vy}{N}\right)}F_{i}[u,v]\quad i=1,2  
\label{式4_24}
\end{equation}
\end{linenomath}
构造完图像的四元数傅里叶变换转换到频域后,我们就可以通过幅度谱分析进行高斯核尺度选择,抑制图像的背景信息,从而突出显著目标。

%-------------------------------------------------------------------------------------------------------------
\subsection{自适应尺度选择}
\label{4_2_2}

本文在第~\ref{4_1_3}节已经分析了幅度谱最优滤波尺度与显著区域尺寸之间具有特定的对应关系,对于一幅图像,针对不同大小显著目标的区域对应一个最佳尺度,但如果要同时获取不同大小显著区域就需要对多个尺度获得的显著图进行融合。我们的思路结构如图~\ref{图4_12}所示,图中$I$为输入的原始图像,图像中含有两个大小不同的显著目标。首先通过一定的方法获得显著目标的大小,例子中大目标$R_{1}$的尺寸为$a\times b$,小目标$R_{2}$的尺寸为$c\times d$,由公式~\ref{式4_9}可以确定显著目标所对应幅度谱滤波尺度$K_{1}$和$K_{2}$,经过频域处理和变换后,可以分别得到全局显著图$M_{1}$和$M_{2}$,最后通过一定的融合机制得到最终的显著图$M$。
\begin{figure}[h]
  \centering
  \includegraphics[height=5.5cm]{图4_12}
  \caption{不同大小尺寸显著目标的显著图获取示意图}   
  \label{图4_12} 
\end{figure}

因此,为了得到图像中每个显著目标所对应的最优尺度,我们应该首先获得显著目标的尺寸。侯晓迪等人根据稀疏性理论,提出了一种简单有效的图像描述子---图像签名(Image Signature)~\cite{HouXiaodiTPAMI2012Signature},该描述子可以用来估测场景中图像的前景(显著目标)和位置。有关图像签名的相关内容,具体可参看第~\ref{3_5}节内容。基于图像签名算子的特点,我们将把它进行扩展,用来估测前景(显著区域)的尺寸和位置。例如,图~\ref{图4_13}中有两个不同尺寸的前景物体,~\ref{图4_13}(b)是通过公式~\ref{式3_30}和~\ref{式3_31}计算signature后反变换的图像,从反变换后的图像我们可以看出图像的噪声比较多,可以利用高斯滤波器对图像进行低通滤波(在我们实验中,高斯滤波器尺度取为13.2),消除噪声。为了能够获得显著目标的大小,我们需要对反变换后的二值图像用最小外接矩形算法将目标的尺寸表示出来,通过分析,我们选择了最大类间方差法,该方法是一种通过自适应阈值确定分割的算法,按照图像的灰度特性,将图像分成目标和背景两个类,如果两个类之间的方差最大,则该阈值就是最佳的阈值。阈值分割后通过最小外界矩形计算出图像的尺寸大小(如图~\ref{图4_13}(c)的红色框),即得到图像的宽度$W$和高度$h$,而且我们通过矩形的中心位置$(m,n)$表示显著目标的位置。
\begin{figure}[h]
  \centering%
  \begin{subfigure}{3cm}
    \includegraphics[height=2.5cm]{图4_13_a}
    \caption{}
  \end{subfigure}
  \hspace{4em}%
  \begin{subfigure}{0.2\textwidth}
    \includegraphics[height=2.5cm]{图4_13_b}
    \caption{}
  \end{subfigure}
  \hspace{4em}%
  \begin{subfigure}{0.25\textwidth}
    \includegraphics[height=2.5cm]{图4_13_c}
    \caption{}
  \end{subfigure}
  \caption{通过图像签名确定不同尺寸和位置的两个显著目标图像的示意图}
  \label{图4_13}
\end{figure}

通过公式~\ref{式4_9}中关于$h×w$函数$f$、第$k$个显著目标尺寸$(h_{k},w_{k})$、图像的尺寸$(H,W)$及$\alpha=0.5$,我们可以计算出对应于第$k$个显著目标的最优尺度$\sigma_{k}$。因此,将图像的幅度谱$A(u,v)$与最优尺度为$\sigma_{k}$高斯核为$g(u,v;\sigma_{k})$的滤波器进行卷积,得到平滑后的幅度谱$\tilde{A}$: 
\begin{linenomath}
\begin{equation}
\tilde{A}_k(u,v)=g(u,v;\sigma_k)\ast A(u,v)
\label{式4_25}
\end{equation}
\end{linenomath}
然后,通过傅里叶反变换得到对应最优尺度为$\sigma_{k}$的显著图$S_{k}$:
\begin{linenomath}
\begin{equation}
S_k=Q^{-1}\left\{\tilde{A}_k(u,v)e^{\mu(u,v)P(u,v)}\right\}
\label{式4_26}
\end{equation}
\end{linenomath}
最后,通过同样的方式求出一幅图像中所有显著目标对应的所有的显著图。

%-------------------------------------------------------------------------------------------------------------
\subsection{自适应权重显著度融合}
\label{4_2_3}

我们通过自适应尺度选择的方法利用不同的自适应滤波器得到不同的显著图,这些显著图最终会整合起来合并为一个显著图,从而得到对同一个场景显著目标更加全面、更加可靠、更加准确的图像描述。

图像融合分为三个层次:像素级的图像融合、特征级的图像融合和决策级的图像融合~\cite{吕超峰2007融合}。基于这些层次,具体的图像融合方法包括很多,如加权平均方法、主成分分析方法、小波变换方法和IHS变换方法等。在视觉系统中,并行处理的视觉信息最终会在视觉皮层中整合在一起,视觉注意的融合也是基于该种机制。过去的一些研究~\cite{JuddICCV2009Learning}已经证明了图像中越接近中央的区域越能吸引人的注意,这表明了靠近图像中心的区域要比远离图像中心的区域更显著,即越靠近图像中心越显著。这种机制可以用中央偏见来描述,中央偏见可以简单有效的用高斯分布来表示。

假设一幅图像的中心为$(m_{c},n_{c})$,第$k$个显著目标的中心为$(m_{k},n_{k})$,则第$k$个显著目标的中央偏见权重$w_{k}$用高斯函数表示为:
\begin{linenomath}
\begin{equation}
\omega_k=e^{-\frac{(m_{c}-m_{k})^{2}+(n_{c}-n_{k})^{2}}{\eta ^{2}}}
\label{式4_27}
\end{equation}
\end{linenomath}
其中,$\eta$为一参数,在我们的实验中被设置为16。因此,最终的显著图$M$表示为:
\begin{linenomath}
\begin{equation}
M=g\ast \left[\frac{\sum_{k=1}^{K}\omega_{k}\cdot S_{k}^{2}}{\sum_{k=1}^{K}\omega_{k}}\right]
\label{式4_28}
\end{equation}
\end{linenomath}
其中,$k$表示所有显著目标的个数,$w_{k}$是通过公式~\ref{式4_27}计算得到的自适应高斯权重值,对应于公式~\ref{式4_26}获得的第$k$个显著图,$g$是一个高斯滤波器,用来提高显著效果。

%-------------------------------------------------------------------------------------------------------------
\subsection{实验结果与分析}
\label{4_2_4}

在本文中,我们将我们的方法与其他频域显著性检测算法进行了结果的对比,我们选择了四个当今国际上比较具有代表性和影响力的静态图像数据集:MSRA数据集~\cite{LiuTieCVPR2007Learning}、ECSSD数据集~\cite{YanQiongCVPR2013Hierarchical}、DUT-OMRON数据集~\cite{YangChuanCVPR2013Manifold}和PASCAL-S数据集~\cite{LiYinCVPR2014Secrets}。

MSRA数据集是最早用于检测大尺度显著目标的数据集,其中包括5,000幅自然图像,数据集中图像的目标都比较明确且单一,如花、水果、动物、门和一些户外风景等。MSRA数据集ground-truth起初是人工标记的矩形框图,后来由Jiang等人~\cite{JiangHuaizuCVPR2013Discriminative}按像素将图像的显著目标分割出来,该数据集显著目标比较大而明显,并且其背景都相对比较简单,在显著性检测的实验中比较容易。DUT-OMRON数据集\cite{YangChuanCVPR2013Manifold}包含5,168幅高质量的图像,它们是从140,000张自然图像人工选择出来的,该数据集的图像包含一个或多个显著目标,并且背景相对比较复杂。DUT-OMRON数据集~\cite{YangChuanCVPR2013Manifold}通过人工标记有矩形框的形式,有像素标记的形式,也有人眼跟踪仪获得的注意视点的标记形式,因此其标记比较全面,不仅可以用来进行显著目标检测,还可以进行注意焦点预测的检测。ECSSD数据集~\cite{YanQiongCVPR2013Hierarchical}是从CSSD数据集中扩充的,该数据集包含1,000个语义上比较有意义但结构比较复杂的图像。DUT-OMRON数据集~\cite{YangChuanCVPR2013Manifold}和ECSSD数据集~\cite{YanQiongCVPR2013Hierarchical}在显著目标检测任务中更加具有挑战性。PASCAL-S数据集~\cite{LiYinCVPR2014Secrets}包含850幅非常复杂的自然图像,图像中包含多个显著目标,它是目前最新的也是非常具有挑战的数据集。

我们选择了7个显著性检测模型与我们的方法进行定性定量分析,这些模型包括:SR模型~\cite{HouXiaodiCVPR2007Residual},PQFT模型~\cite{GuoChenleiCVPR2008Spatio},PFDN模型~\cite{BianCognNeurodyn2010Visual},FT模型~\cite{AchantaCVPR2009Frequency},SIG模型~\cite{HouXiaodiTPAMI2012Signature},Wavelet模型~\cite{ImamogluTMM2013wavelet},和HFT模型~\cite{LiJianTPAMI2013Scale}。我们选择这些模型的原因是它们的检测都是在频域进行处理的(谱显著性检测模型)。

在定性定量分析对比方面,我们根据Achanta等人~\cite{AchantaCVPR2009Frequency}提出的评价标准与选择的7种显著性检测模型进行了对比。首先我们在上述的数据集上分别用8种方法进行处理,从显著图上提取出显著目标,最基本简单的方法就是全局阈值分割方法~\cite{王岩2012注意},我们把阈值从0逐步增加到255,利用每一个分割的阈值对得到的显著图进行分割,计算对应阈值下的精度和召回率,然后将精度和召回率在数据集上进行平均,画出了精度--召回率曲线(p--R曲线),见图~\ref{图4_14}和~\ref{图4_15}。从图中所给的结果可以看到,我们的方法获得的精度--召回率曲线都位于其他方法获得的曲线的上面,我们的精度、召回率值更加接近于$(1,1)$点的坐标(有关精度--召回率曲线的具体介绍和分析见~\ref{2_5_1})。因此,本文所提的方法较其他的频域方法在显著目标检测方面相比,我们的方法更好,尤其是在具有复杂背景和显著目标较多的数据集ECSSD~\cite{YanQiongCVPR2013Hierarchical}和DUT--OMRON~\cite{YangChuanCVPR2013Manifold}上,我们的方法得到的PR值仍是最好的,在最新的数据集PASCAL--S数据集~\cite{LiYinCVPR2014Secrets}上,我们的结果也是最高的。从精度--召回率曲线上我们还可以看出,SR和PQFT结果值比较接近,这与我们之前分析谱剩余算法与相位谱算法十分接近是一致的,出现小的差距的原因(P--R曲线PQFT结果要优于SR方法)是PQFT模型是在相位谱算法的基础上提取了图像的多个底层特征(RG,BY,I等),而SR方法只运用了亮度特征。
\begin{figure}[h]
  \centering%
  \begin{subfigure}{5.5cm}
    \includegraphics[height=10cm]{图4_14_a.eps}
    \caption{}
  \end{subfigure}
  \hspace{4em}%
  \begin{subfigure}{0.45\textwidth}
    \includegraphics[height=10cm]{图4_14_b.eps}
    \caption{}
  \end{subfigure}
  \caption{精度--召回率曲线和F--测量值。从左到右分别为MSRA数据集和ECSSD数据集; 从上到下分别是不同方法在对应数据集上测得的结果,上面是精度--召回率曲线,下面是F--测量值。}
  \label{图4_14}
\end{figure}
\begin{figure}[h]
  \centering%
  \begin{subfigure}{5.5cm}
    \includegraphics[height=10cm]{图4_15_a.eps}
    \caption{}
  \end{subfigure}
  \hspace{4em}%
  \begin{subfigure}{0.45\textwidth}
    \includegraphics[height=10cm]{图4_15_b.eps}
    \caption{}
  \end{subfigure}
  \caption{精度--召回率曲线和F--测量值。从左到右分别为DUT-OMRON数据集和PASCAL-S数据集; 从上到下分别是不同方法在对应数据集上测得的结果,上面是精度--召回率曲线,下面是利用自适应阈值分割获得的F--测量值。}
  \label{图4_15}
\end{figure}

另外,我们为了更加综合、更加全面、更加准确的说明显著目标检测的好坏,引入了一种综合性的显著性检测评价指标---F--测量(F--measure)。F--测量的定义为:
\begin{linenomath}
\begin{equation}
F_{\beta}(T)=\frac{(1+\beta^{2})\cdot Precision(T)\cdot Recall(T)}{\beta^{2}\cdot Precision(T)+Recall(T)}
\label{式4_29}
\end{equation}
\end{linenomath}
正如之前的方法~\cite{AchantaCVPR2009Frequency,YanQiongCVPR2013Hierarchical,ShenXiaohuiCVPR2012unified}一样,我们在实验中也将$\beta^{2}$设置为0.3,$T$是一个自适应分割值,定义为获得的显著图平均显著度的二倍:
\begin{linenomath}
\begin{equation}
T=\min\left\{2\times \frac{\sum^{M}_{i}S(I_{i})}{M},T_{\max}\right\}
\label{式4_30}
\end{equation}
\end{linenomath}
其中,$M$代表显著图中像素的数量,$i$是图像像素索引,$T_max$代表$T$的最大上限值,在我们的实验中被设置为255。图~\ref{图4_14}和图~\ref{图4_15}底面的直方图即我们测得的结果值,从获得F--测量的结果看,我们的方法得到的综合指标值是最高的,所以,在F--测量评价标准上,我们的算法相对于所选择的算法来说也是最好的。

为了更加直观的对比我们的方法和其他算法的优劣,我们从每个数据集上随机选择了一些显著图~\ref{图4_16},其中A是MSRA数据集~\cite{LiuTieCVPR2007Learning}的图像,B是ECSSD数据集~\cite{YanQiongCVPR2013Hierarchical},C是DUT-OMRON数据集~\cite{YangChuanCVPR2013Manifold},D是PASCAL-S数据集~\cite{LiYinCVPR2014Secrets}。从这些图像中我们可以看出,A1、A2、B1、B2、C1、C2、D1和D2包含不同大小的单个显著目标,它们的背景有的简单、有的复杂; A3、B3、C3和D3中的每一幅图像中都包含两个或多个显著物体,这些不同的显著物体不一样; A3、B1和C1的显著目标的尺寸比较小,而其他图像中的显著目标的尺寸相对比较大。从所给的对比显著图的结果我们可以看出,SR模型~\cite{HouXiaodiCVPR2007Residual}和PQFT模型~\cite{GuoChenleiCVPR2008Spatio}对于大的显著目标只能突出目标的边缘或纹理密集的区域,对于小的显著目标可以检测出来,因此,大目标的显著性检测数据集对于这两种算法来说比较具有挑战性; PFDN算法~\cite{BianCognNeurodyn2010Visual}和SIG算法~\cite{HouXiaodiTPAMI2012Signature}突显的显著目标太模糊; FT~\cite{AchantaCVPR2009Frequency}和Wavelet算法~\cite{ImamogluTMM2013wavelet}在检测复杂结构的图像时是没有效果的; HFT算法~\cite{LiJianTPAMI2013Scale}可以突出显著目标,但是突出的不够均匀,并且尤其对于检测多个显著目标的情况下还会丢失显著性信息; 我们提出的算法对于大的显著目标和小的显著目标都可以检测,并且可以均匀的突出整个显著区域,显著边缘比较清晰,因此我们的算法更加合理、更加有效。
\begin{figure}[h]
  \centering
  \includegraphics[height=13.5cm]{图4_16}
  \caption{不同算法在不同数据集上的显著性检测结果}   
  \label{图4_16} 
\end{figure}

%-------------------------------------------------------------------------------------------------------------
\subsection{本章小节}
\label{4_2_5}

本章我们提出了一种谱计算模型的自底向上的图像显著目标检测算法---基于幅度谱分析的自适应显著目标检测方法。我们利用显著目标的尺寸与幅度谱滤波尺度的关系找到每个显著目标所对应的最优尺度,然后对幅度谱进行平滑滤波处理,抑制图像的背景,从而突出图像显著性区域。另外,我们的算法可以自适应的融合不同尺度高斯滤波器滤波处理得到的显著图,这些显著图包含不同目标所对应的显著性信息。考虑到人们的视觉注意大多集中在图像的中心区域,我们引入了高斯分布来描述这种中央偏见机制。为了证明方法的有效性和可靠性,我们在国际上公开的数据集上定性定量比较了各种方法,实验效果证明了我们所提出的方法要比其他谱显著性检测算法更加精确、更加有效。

















