
%%% Local Variables: 
%%% mode: latex
%%% TeX-master: t
%%% End: 

\chapter{频域显著性检测}
\label{cha2}

随着计算机技术的进步和发展,图像显著性检测的要求越来越高。频域显著性检测技术适应于提高计算的准确度和效率,降低计算的复杂度,避免不必要的资源浪费。图像由空间域转换到频域进行处理必须进行适当的时频转换,如傅里叶变换、小波变换等。不管图像是怎样的,图像的空间域信息在转换到频域时都被保留在幅度谱和相位谱中,当它们之间进行相互转换时信息不会发生丢失。频域显著性检测一般是将输入的图像进行前期处理,如将图像的转换到不同颜色空间来提取各种特征、图像尺度变换等; 然后进行时频转换把图像转换到频域,得到图像相应的谱成分; 再对相应的谱进行一定的处理得到处理后的谱; 然后通过反变换将处理后的谱结合把图像由频域转换到空间域;最后,经过图像的后续处理得到最终的显著图。频域显著性检测的一般步骤见图~\ref{图2_1}。接下来的内容将针对频域显著性检测的各个阶段进行归纳和综述。

\begin{figure}[h]
  \centering
  \includegraphics[height=0.8cm]{图2_1}
  \caption{频域显著性检测的一般步骤}
  \label{图2_1}    
\end{figure}

%============================================================================================================
\section{图像频域显著性检测前期处理}
\label{2_1}

大多数频域显著性检测模型在进行处理时,首先要进行图像的前期处理,包括特定颜色空间转换用来提取特征、尺度变换等。接下来,我们将对图像的彩色空间和尺度变换进行总结。

%-------------------------------------------------------------------------------------------------------------
\subsection{RGB彩色模型}
\label{2_1_1}

RGB彩色空间是目前运用最广的颜色模型之一,是工业界的一种颜色标准,被大多数人所熟识。它是通过对红、绿、蓝三原色亮度的变化及他们相互之间的叠加得到各种各样的颜色的,该标准几乎包含了人类视觉所感知的所有颜色。

\begin{figure}[t]
  \centering
  \includegraphics[height=5cm]{图2_2}
  \caption{RGB彩色模型}
  \label{图2_2}    
\end{figure}

图~\ref{图2_2}所示的即为RGB彩色模型对应的坐标系\cite{ZhangZheng2010book}。在立方体的3个顶点上分别是红、绿、蓝,青、黄和深红分别位于另外的3个顶点上,黑色位于坐标原点,白色则在离原点最远的顶点上,灰度级沿着这两点的连线依次分布,不同的颜色位于立方体的内部和外部,因此所有的颜色都可以通过一个3维向量进行表示。例如,假设将所有颜色归一化为$[0,1]$,红色可以表示为$(1,0,0)$,白色可以表示为$(1,1,1)$。

在Matlab实验中,我们可以直接将图像的R、G、B颜色特征提取出来,如图~\ref{图2_3},RGB彩色空间常用于图像/视频采集、图像表达和图像显示等。

\begin{figure}[b]
\begin{minipage}{0.24\textwidth}
  \centering
  \includegraphics[height=2cm]{图2_3_a}
  \label{图2_3_a}
\end{minipage}\hfill
\begin{minipage}{0.24\textwidth}
  \centering
  \includegraphics[height=2cm]{图2_3_b}
  \label{图2_3_b}
\end{minipage}
\begin{minipage}{0.24\textwidth}
  \centering
  \includegraphics[height=2cm]{图2_3_c}
  \label{图2_3_c}
\end{minipage}
\begin{minipage}{0.24\textwidth}
  \centering
  \includegraphics[height=2cm]{图2_3_d}
  \label{图2_3_d}
\end{minipage}
  \caption{图像的RGB颜色模型例子。从左至右分别为原图像、R通道图像、G通道图像和B通道图像。}
  \label{图2_3} 
\end{figure}

%-------------------------------------------------------------------------------------------------------------
\subsection{Lab彩色模型}
\label{2_1_2}

Lab模型是由CIE(国际照明委员会)制定的一种彩色模式,自然界中的任何色彩都可在Lab空间表达出来\cite{ZhangZheng2010book}。Lab颜色被设计来接近人类视觉,该空间中的数值表示正常实力的人能够看到的所有颜色。Lab描述的是颜色的显示方式,而不是生成颜色所需要的特定色料的数量,因此,它被看作和设备无关的颜色模型。

Lab模型是有亮度L和有关彩色a和b三个部分组成。L代表亮度,a代表从红色/品红色到绿色的部分,b代表从黄色到蓝色的部分。L的值从0至100,分别由纯黑转为纯白,a和b的取值范围都是$[127,-128]$,计算表达式如下:
\begin{linenomath}
\begin{align}
L^{*} &= 116f\big(\frac{Y}{y_{n}}-16\big)\label{式2_1}\\
a^{*} &= 500\big[f(\frac{X}{X_{n}})-f(\frac{Y}{Y_{n}})\big]\label{式2_2}\\
b^{*} &= 200\big[f(\frac{Y}{Y_{n}})-f(\frac{Z}{Z_{n}})\big]\label{式2_3}\\
y &= \left\{ \begin{array}{ll}
t^{\frac{1}{3}}, & \textrm{if $t>(\frac{6}{29})^{3}$}\\
\frac{1}{3}(\frac{29}{6})^{2}t+\frac{4}{29}, & \textrm{if $t\leq (\frac{6}{29})^{3}$}
\end{array} \right.\label{式2_4}\\
f(t) &= \left\{ \begin{array}{ll}
t^{\frac{1}{3}}, & \textrm{if $t>(\frac{6}{29})^{3}$}\\
\frac{1}{3}(\frac{29}{6})^{2}t+\frac{4}{29}, & \textrm{if $t\leq (\frac{6}{29})^{3}$}
\end{array} \right.\label{式2_5}
\end{align}
\end{linenomath}
其中,$X_{n}$,$Y_{n}$和$Z_{n}$分别是CIE XYZ彩色空间的三个值,图~\ref{图2_4}即为Lab彩色空间图,Lab彩色空间常用于颜色光适应、图像压缩和色彩差度量等。
\begin{figure}[h]
  \centering
  \includegraphics[height=5cm]{图2_4}
  \caption{Lab彩色空间图}
  \label{图2_4}    
\end{figure}

%-------------------------------------------------------------------------------------------------------------
\subsection{IRGBY彩色模型}
\label{2_1_3}

在人类视觉系统中,颜色主要有颜色拮抗系统进行表示,即神经细胞对颜色的处理具有双接抗的性质,一种颜色可以使细胞产生兴奋,它所对应的拮抗颜色可以抑制该细胞的兴奋\cite{Engel1997Colour}。经过研究发现,视皮层中有四种彩色拮抗对,分别是红绿、绿红、蓝黄和黄蓝,因此,IRGBY彩色模型在图像处理中的应用也非常多,如显著性检测、图像/视频压缩等。IRGBY颜色空间定义如下:假设$r$,$g$和$b$分别代表红、绿、蓝三种原色,四个颜色特征分别定义为:
\begin{linenomath}
\begin{align}
\textrm{Red:$R=r-\frac{g+b}{2}$,}\label{式2_6}\\
\textrm{Green:$G=g-\frac{r+b}{2}$,}\label{式2_7}\\
\textrm{Blue:$B=b-\frac{r+g}{2}$,}\label{式2_8}\\
\textrm{Yellow:$Y=\frac{r+g}{2}-\frac{|r+g|}{2}-b$.}\label{式2_9}
\end{align}
\end{linenomath}
亮度通道$I$是红、绿、蓝三原色的平均,在显著性检测中,我们只需对拮抗颜色取绝对值就可以表示出两个拮抗颜色中相对兴奋的那一种。因此,红绿和绿红可以用一个公式表示,蓝黄和黄蓝可以用一个公式表示:
\begin{linenomath}
\begin{align}
RG &= \big|R-G\big|\label{式2_10}\\
BY &= \big|B-Y\big|\label{式2_11}\\
I &= \frac{r+g+b}{3}\label{式2_12}
\end{align}
\end{linenomath}

%-------------------------------------------------------------------------------------------------------------
\subsection{图像分辨率尺度调整}
\label{2_1_4}

在显著性检测处理初期,许多模型都会首先将图像进行分辨率调整。例如,原图像为$500×500$像素的图像,经过调整后变为$64×64$大小的图片,这样做的目的\cite{ZhangLiming2010Book}包括两个方面:一是为了加快图像的处理速度,自底向上的显著性检测只是将图像显著部分突出出来,并不会太注重图像的细节,因此不会对图像处理效果产生必要的影响; 另外,调整图像分辨率可以将图像高频区域的噪声抑制掉,从而更有利用显著性的检测。标准图像$I$分辨率尺度转换的步骤为:
\begin{equation}
I'=\textrm{resize$(I)$}
\label{式2_13}
\end{equation}

%\subsection{各种频域显著性检测模型前期处理方法}
%\label{subsec:fifth}
%-------------------------------------------------------------------------------------------------------------
%\subsection{本章小节}
%\label{sec:fifth}

%本章详细介绍了频域显著性检测算法前期处理的各种方法。这些方法包括彩色模型的建立用来提取图像的相关特征,图像分辨率尺度变换用来加快图像处理速度以及抑制图像噪声。其中,彩色模型的建立包括RGB彩色模型、Lab彩色模型和IRGBY彩色模型。本章对这些方法的原理和步骤进行了介绍。

%=============================================================================================================
\section{频域显著性检测时频变换}
\label{2_2}

频域显著性检测主要是利用图像的空间域信息在频域进行处理。在大多数情况下,空间域处理和频域处理可以看作对图像显著性检测问题殊途同归的两种解决方式。而在另一些情况下,有些显著性检测问题更适合在频域中完成。频域显著性检测的优势在于:一方面可以加快图像显著性检测的速度、提高检测效率,有利于图像进行实时性处理; 另一方面,频域处理为显著性检测提供了新的思路,通过分析图像显著性区域在频域幅度谱和相位谱的分布特点,获得显著性检测的新方法。要进行频域处理,就需要对图像进行时频的相互转换。图像的时频变换包括傅里叶变换,离散余弦变换以及他们的扩展包括,四元数傅里叶变换和四元数离散余弦变换,小波变换等。下面将分别介绍图像时频变换。

%-------------------------------------------------------------------------------------------------------------
\subsection{傅里叶变换}
\label{2_2_1}

法国数学家傅里叶发现任何周期函数只要满足一定的条件(狄利赫里条件)都可以用正弦和余弦函数的加权和来表示\cite{ZhangZheng2010book}。傅立叶变换提供了一种变换到频率域的手段,通过用用傅里叶变换表示的函数特征可以完全利用傅里叶反变换进行重建,不容易丢失信息。

一维函数$f(x)$(其中$-\propto<x<\propto$)的傅里叶变换为:
\begin{linenomath}
\begin{align}
F(u)=\int^{\propto}_{-\propto}f(x)e^{-i2\pi ux}dx
\label{式2_14}
\end{align}
\end{linenomath}
根据$F(u)$我们可以通过傅里叶反变换的到$f(x)$:
\begin{linenomath}
\begin{align}
f(x)=\int^{\propto}_{-\propto}F(u)e^{i2\pi ux}du
\label{式2_15}
\end{align}
\end{linenomath}
上述两个式子就是我们通常提到的傅里叶变换对。对于一维函数$f(x)$(其中$x=0,1,2,\dots,M-1$)的傅里叶变换的离散形式为:
\begin{linenomath}
\begin{align}
F(u)=\sum^{M-1}_{x=0}f(x)e^{-i2\pi ux/M}, u=0,1,2,\dots,M-1
\label{式2_16}
\end{align}
\end{linenomath}
相应的反变换为:
\begin{linenomath}
\begin{align}
f(x)=\frac{1}{M}\sum^{M-1}_{x=0}F(u)e^{i2\pi ux/M},u=0,1,2,\dots,M-1
\label{式2_17}
\end{align}
\end{linenomath}
有了一维的基础,连续傅里叶变换及其反变换推广到二维分别为:
\begin{linenomath}
\begin{align}
F(u,v) &= \int^{\propto}_{-\propto}\int^{\propto}_{-\propto}f(x,y)e^{-i2\pi (ux+vy)}dxdy\label{式2_18}\\
f(x,y) &= \int^{\propto}_{-\propto}\int^{\propto}_{-\propto}F(u,v)e^{i2\pi (ux+vy)}dudv\label{式2_19}
\end{align}
\end{linenomath}
而在数字图像处理中,我们研究的是二维离散函数的傅里叶变换,则二维离散傅里叶变换及反变换公式为:
\begin{linenomath}
\begin{align}
F(u,v) &= \sum^{M-1}_{x=0}\sum^{N-1}_{y=0}f(x,y)e^{-j2\pi (ux+vy)}dxdy\label{式2_20}\\
f(x,y) &= \sum^{M-1}_{u=0}\sum^{N-1}_{v=0}F(u,v)e^{j2\pi (ux+vy)}dudv\label{式2_21}
\end{align}
\end{linenomath}
相对于空间域的变量x、y,这里的u、v则是变换域或者说是频率域的变量。傅里叶变换的极坐标的表示形式为:
\begin{equation}
F(u,v)=\Vert F(u,v)\Vert e^{j\phi(u,v)}
\label{式2_22}
\end{equation}
其中,$\Vert F(u,v)\Vert$代表图像的幅度谱,$\phi(u,v)$代表图像的相位谱。图像的频率表征了图像中灰度变化的剧烈程度,图~\ref{图2_5}展示了图像傅里叶变换的效果图。
\begin{figure}[t]
  \centering%
  \begin{subfigure}{3cm}
    \includegraphics[height=3cm]{图2_5图2_6_a}
%    \caption{原图像}
  \end{subfigure}
  \hspace{4em}%
  \begin{subfigure}{0.25\textwidth}
    \includegraphics[height=3cm]{图2_5_b}
%    \caption{图像的离散余弦变换}
  \end{subfigure}
  \caption{图像的傅里叶变换图例}
  \label{图2_5} 
\end{figure}

二维傅里叶变换图像信号能量将集中在矩阵的四个角上,例子中的频谱是经过平移之后的图像,平移之后中间部分是低频,低频亮度越大说明能量越大,对于图像而言,能量大多数集中在低频区域,而且能量越低越稳定\cite{Gonzalez2005book}。值得说明的是,在显著性检测算法中,尤其计算尺寸较大的图像,我们为了提高计算效率,因此会选择傅里叶变换的快速形式,即快速傅里叶变换(FFT)用来获得图像的离散频域谱。有关图像频域幅度谱和相位谱我们将在第\ref{2_3}节进行具体的分析和介绍。

%-------------------------------------------------------------------------------------------------------------
\subsection{离散余弦变换}
\label{2_2_2}

根据上一节我们知道,离散傅里叶变换是复数运算,实偶函数的傅里叶变换只含实的余弦项,于是构造了一种实数域的变换--离散余弦变换(DCT)\cite{Rao2014book}。离散余弦变换相当于一个长度大概是它两倍的离散傅里叶变换,除了具有一般的正交变换性质外,其变换阵的基向量很近似于Toeplitz矩阵的特征向量,大多数自然信号的能量都集中在离散余弦变换之后的低频部分,是语音、图像信号变换最佳正交变换。目前,在一系列视频压缩编码的国际标准建议中是一个基本的处理模块。

对于一幅图像$M×N$,二维离散余弦变换的表示形式为:
\begin{linenomath}
\begin{align}
F(0,0) &= \frac{1}{\sqrt{MN}}\sum_{x=0}^{M-1}\sum_{y=0}^{N-1}f(x,y)\label{式2_23}\\
F(0,v) &= \frac{\sqrt{2}}{\sqrt{MN}}\sum_{x=0}^{M-1}\sum_{y=0}^{N-1}f(x,y)cos\frac{(2y+1)v\pi}{2N}\label{式2_24}\\
F(u,0) &= \frac{\sqrt{2}}{\sqrt{MN}}\sum_{x=0}^{M-1}\sum_{y=0}^{N-1}f(x,y)cos\frac{(2x+1)u\pi}{2N}\label{式2_25}\\
F(u,v) &= \frac{2}{\sqrt{MN}}\sum_{x=0}^{M-1}\sum_{y=0}^{N-1}f(x,y)cos\frac{(2x+1)u\pi}{2N}cos\frac{(2y+1)v\pi}{2N}\label{式2_26}
\end{align}
\end{linenomath}
二维离散余弦反变换形式为:
\begin{linenomath}
\begin{align}
\lefteqn{ f(x,y)=\frac{1}{\sqrt{MN}}F(0,0)+\frac{\sqrt{2}}{\sqrt{MN}}\sum_{v=1}^{N-1}F(0,v)cos\frac{(2y+1)v\pi}{2N}+\frac{\sqrt{2}}{\sqrt{MN}}\sum_{u=1}^{M-1}F(u,0){} }\nonumber\\
& & {}cos\frac{(2x+1)v\pi}{2N}+\frac{2}{\sqrt{MN}}F(u,v)cos\frac{(2x+1)v\pi}{2N}cos\frac{(2y+1)v\pi}{2N}
\label{式2_27}
\end{align}
\end{linenomath}

\begin{figure}[t]
  \centering%
  \begin{subfigure}{3cm}
    \includegraphics[height=3cm]{图2_5图2_6_a}
%    \caption{原图像}
  \end{subfigure}
  \hspace{4em}%
  \begin{subfigure}{0.25\textwidth}
    \includegraphics[height=3cm]{图2_6_b}
%    \caption{图像的离散余弦变换}
  \end{subfigure}
  \caption{图像的离散余弦变换图例}
  \label{图2_6}
\end{figure}

二维离散余弦变换具有系数为实数,正变换与逆变换的核相同的特点。由于离散余弦变换DCT信息强度集中,图像在进行DCT变换后(如图~\ref{图2_6}),在频域中矩阵左上角低频的幅度值大而右下角高频幅值较小,进行量化后会产生大量的零值系数,因此,在编码时可以压缩数据\cite{Rao2014book},被广泛用于视频编码图像压缩,在显著性检测中也逐渐被使用。

%-------------------------------------------------------------------------------------------------------------
\subsection{小波变换}
\label{2_2_3}

尽管小波变换(WT)在20世纪早期被Alfred Haar首次提出,但是到20世纪后期这个领域才得到进步\cite{Merry2013wavelet}。小波变换可以解决傅里叶变换不能解决的许多问题,目前被广泛应用在许多领域,如信号去噪、图像增强、数据压缩、视频编码、模式分类等\cite{ImamogluTMM2013wavelet}。

傅里叶变换是时频之间转换的工具,实质上是将时域(空间域)的信号分解成许多不同频率的正弦波的叠加。虽然傅里叶变换将信号的时域(空间域)与频域特征进行了联系,可以从信号的时域(空间域)和频域分别进行分析,但我们却很难将时频结合起来进行观察,这是因为通过傅里叶变换将整个时间域进行积分,得到的傅里叶谱是信号的统计特征,不能局部化分析信号。因此,利用傅里叶变换对信号分析将会面临一个矛盾,即时域和频域局部化矛盾\cite{Merry2013wavelet,Fugal2009book,Kocyigit2013EMG,Semmlow2004book}。为了解决这种矛盾,Dennis Gabor引入了短时傅里叶变换(STFT),进而对傅里叶变换进行了推广,其基本思想是将信号分成许多间隔,利用傅里叶变换分析每一段的间隔,从而计算在该间隔的频率。STFT的窗的宽度对于观察所有的频率是固定不变的,如果对于高频信号,在一段较长的窗内,可能会有很多周期,求出的傅里叶变换的系数将是许多周期的平均值,所以在这种情况下,短时傅里叶变换的局部性能得不到很好的体现,如果时间窗减小,则对于很低的低频信号检测不到。因此,为了更好的解决信号的多分辨率分析问题,小波分析诞生并发展了起来。

多尺度小波分析是时域(空间域)和频域中的局部函数,也类似于窗口函数,通过定义它的时频中心和半径衡量其局部化程度。小波基可以通过改变它的尺度因子是需要分析的信号在高频时时间域分辨率高,低频时频率域分辨率高,从而达到多尺度分析的效果。正交小波滤波器组的一个特性是将信号分解为低频子带和高频子带,它们分别对应于近似信号和细节信号。对于一维3次小波分解如图~\ref{图2_7}所示\cite{Fugal2009book},$s[n]$是覆盖所有频段的一维信号,$cA_{1}$、$cA_{2}$和$cA_{3}$是每一层分解的近似信号,对应于低频子带,$cD_{1}$、$cD_{2}$和$cD_{3}$是每一层分解的细节信号,对应于频域的高频子带。
\begin{figure}[t] % use float package if you want it here
  \centering
  \includegraphics[height=3.5cm]{图2_7}
  \caption{3层小波分解示意图}
  \label{图2_7}
\end{figure}

小波分解在多尺度方面具有提取方向细节(水平、垂直和対角线)的优势\cite{Merry2013wavelet,Semmlow2004book},并且使得空间高频的高分辨率部分和空间低频的低分辨率部分在分解的过程中不会丢失信息,因此,小波分解层数的选取通过忽略近似信号的低频子带在带通区域内通过重构到特征图,如图~\ref{图2_7}我们可以看到,如果去掉近似信号,通过小波反变换获得的信号将会包含信号的边缘和纹理信息,通过不同层的小波分解与重构的处理从而获得不同的边缘和纹理信息,最终得到不同尺度下的特征图。

%------------------------------------------------------------------------------------------------------------------------------
\subsection{四元数傅里叶变换}
\label{2_2_4}

四元数是由英国物理学家 Hamilton 最早提出的概念\cite{Hamilton1866book},它是由普通复数进一步得到的,由于复数可以将多个数值通过复数的形式结合在一起,因此复数已经被广泛的应用在数学、物理和工程中。四元数在进行加法运算规则中与复数具有相似的特点,但是其乘法运算与一般的复数就有很大的差异,这也是两者之间非常重要的区别之一。四元数自提出以后有了很大的发展,很多数学工作都引入了四元数\cite{Kantor1989book}。

四元数是由一个实数和三个虚数组成,其表达式$q$如下所示:
\begin{linenomath}
\begin{align}
q=a+\mu_{1}b+\mu_{2}c+\mu_{3}d
\label{式2_28}
\end{align}
\end{linenomath}
其中,$a$、$b$、$c$和$d$表示实数,$\mu_{1}$、$\mu_{2}$和$\mu_{3}$分别表示三个正交的虚数单位,$a$代表四元数的实数部分,$\mu_{1}b+\mu_{2}c+\mu_{3}d$代表四元数的虚数部分,且$\mu_{1}^{2}=\mu_{2}^{2}=\mu_{3}^{2}=-1$,$\mu_{1}\cdot\mu_{2}=\mu_{3}=-\mu_{2}\cdot\mu_{1}$,$\mu_{2}\cdot\mu_{3}=\mu_{1}=-\mu_{3}\cdot\mu_{2}$,$\mu_{3}\cdot\mu_{1}=\mu_{2}=-\mu_{1}\cdot\mu_{3}$。从上面的运算可以看出,四元数的乘法运算不满足交换率。四元数的共轭表达式$q^{*}=a-\mu_{1}b-\mu_{2}c-\mu_{3}d$,则四元数的模值$\Vert q \Vert$为:
\begin{linenomath}
\begin{align}
\Vert q \Vert=\Vert \sqrt{q^{*}\cdot q} \Vert=\sqrt{a^{2}+b^{2}+c^{2}+d^{2}}
\label{式2_29}
\end{align}
\end{linenomath}
则$q$的极坐标表示形式为:
\begin{linenomath}
\begin{align}
q=\Vert q \Vert e^{\mu \varphi}
\label{式2_30}
\end{align}
\end{linenomath}
其中,$e^{\mu \varphi}=cos\varphi+\mu sin\varphi$,$\mu=(\mu_{1}b+\mu_{2}c+\mu_{3}d)/\sqrt{b^{2}+c^{2}+d^{2}}$,$cos\varphi=a/\Vert q \Vert$,$sin\varphi=\sqrt{b^{2}+c^{2}+d^{2}}/\Vert q \Vert$,且$\varphi=tan^{-1}(\sqrt{b^{2}+c^{2}+d^{2}}/a)$。

假设有两个四元数$q_{1}=a_{1}+\mu_{1}b_{1}+\mu_{2}c_{1}+\mu_{3}d_{1}$和$q_{2}=a_{2}+\mu_{1}b_{2}+\mu_{2}c_{2}+\mu_{3}d_{2}$,则二者之间的加减法运算满足:
\begin{linenomath}
\begin{align}
q_{1}\pm q_{2}=(a_{1}\pm a_{2})+\mu_{1}(b_{1}\pm b_{2})+\mu_{2}(c_{1}\pm c_{2})+\mu_{3}(d_{1}\pm d_{2})
\label{式2_31}
\end{align}
\end{linenomath}
两个四元数的乘法满足:
\begin{linenomath}
\begin{align}
\lefteqn{ q_{1}\cdot q_{2}=(a{1}\cdot a_{2}-b{1}\cdot b_{2}-c{1}\cdot c_{2}-d{1}\cdot d_{2}) {}}\nonumber\\
&& {}+\mu_{1}(a{1}\cdot b_{2}+b{1}\cdot a_{2}+c{1}\cdot d_{2}-d{1}\cdot c_{2})\nonumber\\
&& {}+\mu_{2}(a{1}\cdot c_{2}-b{1}\cdot d_{2}+c{1}\cdot a_{2}+d{1}\cdot b_{2})\nonumber\\
&& {}+\mu_{3}(a{1}\cdot d_{2}+b{1}\cdot c_{2}-c{1}\cdot b_{2}+d{1}\cdot a_{2})
\label{式2_32}
\end{align}
\end{linenomath}

四元数傅里叶变换具有两种形式,包括连续和离散。对于研究的图像问题,我们往往采用离散形式的四元数傅里叶变换(DQFT)。设四元数矩阵$f(m,n)$,$m$和$n$分别代表矩阵的行和列位置,离散四元傅里叶变换后的频域坐标分别用$u$和$v$表示,设$f_{1}(x,y)=a+\mu_{1} b$,$f_{2}(x,y)=c+\mu_{1} d$,则$q=f_{1}(x,y)+\mu_{2}f_{2}(x,y)$。离散四元数傅里叶变换的表达式为:
\begin{linenomath}
\begin{align}
Q[u,v] &= F_{1}[u,v]+F_{2}[u,v]\mu_{2}\label{式2_33}\\
F_{i}[u,v] &= \frac{1}{\sqrt{MN}}\sum_{x=0}^{M-1}\sum_{y=0}^{N-1}e^{-j2\pi (\frac{ux}{M}+\frac{vy}{N})}f_{i}(x,y)  i=1,2\label{式2_34}
\end{align}
\end{linenomath}
其中$M$和$N$分别代表矩阵的大小,离散四元数傅立叶反变换为$q(x,y)=Q^{-1}[u,v]$。四元数傅里叶变换在彩色图像处理中逐渐被人们认识并在多个领域中有了广泛的应用,如彩色图像压缩、彩色图像平滑及彩色图像的显著性检测等,有关图像的四元数傅里叶变换本文将在第\ref{4_2_1}章中进行详细介绍。
%=============================================================================================================
\section{频域显著性检测频域分析与处理}
\label{2_3}

根据信号处理理论可知,一维信号和二维信号在进行变换时很多性质是一致的,但是对于二维图像信号而言,很多人对其频谱没有一个清晰的认识,因此,对于显著性检测频谱处理更不能理解。下面我们将对图像信号的频谱进行分析。

%-------------------------------------------------------------------------------------------------------------
\subsection{图像信号的频谱}
\label{2_3_1}

上一节公式~\ref{式2_22}已经给出了傅里叶变换极坐标的表达形式,$\Vert F(u,v)\Vert$代表图像的幅度谱,$\phi(u,v)$代表图像的相位谱,而幅度谱和相位谱用傅里叶变换后的实虚函数表示分别为:
\begin{linenomath}
\begin{align}
\Vert F(u,v)\Vert &= [Re(u,v)^{2}+Im(u,v)^{2}]^{1/2}\label{式2_35}\\
\phi(u,v) &= arg tan\frac{Im(u,v)}{Re(u,v)}\label{式2_36}
\end{align}
\end{linenomath}
其中,$Re(u,v)$与$Im(u,v)$分别代表$F(u,v)$的实部和虚部。在图像频域中,通过傅里叶变换可以将图像分为幅度谱和相位谱,他们分别代表了不同含义的信息\cite{ZhangRuolan2002frequency}。图~\ref{图2_8}(b)和(c)分别给出了图~\ref{图2_8}(a)中图像的幅度谱和相位谱。
\begin{figure}[h]
  \centering%
  \begin{subfigure}{3cm}
    \includegraphics[height=3cm]{图2_5图2_6_a}
    \caption{}
  \end{subfigure}
  \hspace{4em}%
  \begin{subfigure}{0.2\textwidth}
    \includegraphics[height=3cm]{图2_8_b}
    \caption{}
  \end{subfigure}
  \hspace{4em}%
  \begin{subfigure}{0.25\textwidth}
    \includegraphics[height=3cm]{图2_8_c}
    \caption{}
  \end{subfigure}
  \caption{图像的幅度谱和相位谱。(a)原图像;(b)图像的幅度谱;(c)图像的相位谱}
  \label{图2_8}
\end{figure}

幅度谱又叫频率谱,它展现了一幅图像所对应的频率分布。频域下的每一点$(u,v)$的幅值$\Vert F(u,v)\Vert$用来表示该频率的正(余)弦波在叠加中所占的比例,决定了一幅图像中所包含的各种频率的分量大小,直接可以反应频率信息。相位谱相对幅度谱表面上看不是很直观,看不出它的重要性,它隐含的是傅里叶变换后的实部和虚部之间对应的某种比例关系,与图像的结构有密切的联系,相位谱决定了傅里叶变换的每一个频率分量分布在图像中的什么位置。我们可以大致的理解,图像的明暗对比度信息包含在幅度谱中,而图像的结构信息包含在相位谱中,我们可以通过实验说明二者各自的贡献。图~\ref{图2_9}(b)是~\ref{图2_9}(a)进行傅里叶反变换时,通过将相位谱置零消去相位谱信息,单独保留幅度谱信息重建的图像,从图中我们已完全看不出原图的任何信息; 图~\ref{图2_9}(c)是~\ref{图2_9}(a)进行傅里叶反变换时,通过将幅度谱设置为某一定值消去幅度谱信息,单独保留相位谱信息重建的图像,从图像中我们基本上可以分辨出元图像大体的结构信息。为了进一步说明幅度谱和相位谱的作用,我们可以构造一个简单的例子进行说明。图~\ref{图2_10} (a)、(b)中分别是一张美女的图片和一张羊的图片,我们将这两幅图像的相位谱进行交换,即将美女的幅度谱加上羊的相位谱(如图~\ref{图2_10}(c)),用羊的幅度谱加上美女的相位谱(如图~\ref{图2_10}(d)),通过傅里叶反变换的公式得到重构后的图像,通过这个例子可以发现,经过交换相位谱和反变换后获得的图像信息与其相位谱所对应的信息是一致的。从以上两个简单的示例可以看出相位谱对于信号重建具有十分重要的作用,幅度的大小影响了图像的灰度信息,但是如果丢失了相位信息,整幅图像的结构和完整性则受到了严重的影响,这也是很多图像处理包括显著性检测经常只对图像变换后的幅度谱进行处理而保留图像相位谱的主要原因。
\begin{figure}[h]
  \centering%
  \begin{subfigure}{3cm}
    \includegraphics[height=3cm]{图2_9_a}
    \caption{}
  \end{subfigure}
  \hspace{4em}%
  \begin{subfigure}{0.2\textwidth}
    \includegraphics[height=3cm]{图2_9_b}
    \caption{}
  \end{subfigure}
  \hspace{4em}%
  \begin{subfigure}{0.25\textwidth}
    \includegraphics[height=3cm]{图2_9_c}
    \caption{}
  \end{subfigure}
  \caption{幅度谱和相位谱重建图像。(a)原图像;(b)幅度谱重建图;(c)相位谱重建图}
  \label{图2_9}
\end{figure}
\begin{figure}[b]
  \centering%
  \begin{subfigure}{0.3\textwidth}
    \includegraphics[height=4cm]{图2_10_a}
    \caption{}
  \end{subfigure}
  \hspace{4em}%
  \begin{subfigure}{0.3\textwidth}
    \includegraphics[height=4cm]{图2_10_b}
    \caption{}
  \end{subfigure}
  \hspace{4em}%
  \begin{subfigure}{0.3\textwidth}
    \includegraphics[height=4cm]{图2_10_c}
    \caption{}
  \end{subfigure}
  \hspace{4em}%
  \begin{subfigure}{0.3\textwidth}
    \includegraphics[height=4cm]{图2_10_d}
    \caption{}
  \end{subfigure}
  \caption{幅度谱和相位谱的关系。(a)美女图像;(b)羊的图像;(c)美女幅度谱加羊相位谱; (d)羊幅度谱加美女相位谱}
  \label{图2_10}
\end{figure}

在频域显著性检测中,主要是通过分析幅度谱和相位谱的特点,对谱进行相应的处理,如对幅度谱进行滤波(高斯滤波、中值滤波等),在频域进行调频处理,利用剩余谱假设理论对幅度谱进行处理,或者直接将幅度谱去掉直接保留图像的相位谱进行傅里叶反变换获得图像的显著信息等方法。有关图像频域显著性检测方法将在第~\ref{cha3}章进行详细分析。

%=============================================================================================================
\section{频域显著性检测的后续处理}
\label{2_4}

为了获得较好的显著性检测效果,在产生显著图后还需要进行进一步处理得到最终的显著图\cite{BorjiTIP2013Quantitative}。正如SR模型~\cite{HouXiaodiCVPR2007Residual}所介绍的,通常经过傅立叶反变换后获得的显著图往往需要对图像中的每一个元素进行平方,然后将平方后的图像与一个合适的高斯滤波器进行滤波从而得到最终的显著图,进行平方的目的是提高图像的对比度,进行高斯滤波的目的\cite{ZhangLiming2010Book}一方面为了抑制图像的噪声,另一方面为了提高显著性检测的效果。其表达式如下:
\begin{linenomath}
\begin{align}
SM(x,y)=g(x,y)\ast S(x,y)^{2}
\label{式2_37}
\end{align}
\end{linenomath}
其中,$S(x,y)$为反变换后的图像,$g(x,y)$为一个低通高斯滤波器,图~\ref{图2_11}表示将获得的显著图进行后续高斯滤波处理得到处理后的显著图,从图中可以明显看出,通过后续滤波处理,显著性区域得到了增强,而背景区域收到了抑制。为了进一步提高显著性检测结果,还可以进行其他的处理,如中央偏见设置(center-bias setting)、边缘切割(border cut)等。
\begin{figure}[h] % use float package if you want it here
  \centering
  \includegraphics[height=7.5cm]{图2_11}
  \caption{利用高斯滤波进行显著性检测后续处理}
  \label{图2_11}
\end{figure}

以前的一些文献通过证明认为靠近图像中心的物体更能够引起人们产生视觉注意~\cite{JuddICCV2009Learning},这个研究说明靠近图像中央的区域要比远离图像中央的区域更显著,并通过实验说明了这种显著性分布可以简单有效的模拟为高斯分布~\cite{ZhangLin2013SDSP},这个理论即为中央偏见。在进行高斯偏见设置时,将图像进行高斯权限设置,即由中心向四周位置分布的元素乘以高斯分布中所对应的值,最后进行归一化处理得到显著图。边缘切割(border cut)是指在进行显著性检测后续处理中将图像的边缘设置成一定像素的窄边区域,这样做的目的是当滤波器位于图像的边缘,滤波相应不能够被很好的定义,并且可以避免边界区域收敛速度慢的缺点~\cite{ChengMingMingCVPR2011Global}。

%=============================================================================================================
\section{评价方法}
\label{2_5}

为了测试显著性检测方法的准确性、客观比较各种不同的方法,我们需要对方法进行评价。显著性检测可以分为注视焦点预测和显著性区域检测\cite{LiYinCVPR2014Secrets},针对不同的分类选择不同的评价方法。显著性检测的数据集需要人工标记获得真实数据(ground-truth),1表示显著性区域,0表示非显著性区域,从最终显著图中提取显著目标可以选择全局阈值分割方法,即将阈值从0逐步增加至255,利用没一个阈值对显著图进行分割,或对显著图进行自适应阈值分割处理~\cite{AchantaCVPR2009Frequency},然后将计算得到的显著图进行定性定量分析,选择合适的评价方法,如精度-召回率曲线(PR曲线)、受试者工作特征曲线(ROC曲线)、F-测量值(F-measure值)或AUC值等。下面将介绍几种显著性检测评价方法。

%-------------------------------------------------------------------------------------------------------------
\subsection{PR曲线和ROC曲线}
\label{2_5_1}

对于分类模型的两类问题,输出可以分为阳性或阴性\cite{Powers2007Evaluation}。在双分类器中有4类可能的结果:如果预测的是阳性,真实的也是阳性(TP),那么此时就叫作真阳性; 如果预测的是阳性,而真实的为阴性,则此时就叫做假阳性(FP); 相反,如果预测的和真实的都是阴性,那么这叫作真阴性(TN); 如果预测的为阴性,真实的为阳性,这时候就称作假阴性(FN),我们可以通过下面的表格~\ref{图2_12}来表示:
\begin{figure}[h] % use float package if you want it here
  \centering
  \includegraphics[height=5.5cm]{图2_12}
  \caption{四种分类结果值}
  \label{图2_12}
\end{figure}
在讨论PR曲线和ROC曲线时往往离不开上面的表格。

PR曲线是指查准率-查全率曲线,在PR空间中,$x$坐标为召回率值(recall),y轴代表精度值(precision),二者分别定义为:
\begin{linenomath}
\begin{align}
\textrm{Recall =$\frac{TP}{TP+FN}$}\label{式2_38}\\
\textrm{precision =$\frac{TP}{TP+FP}$}\label{式2_39}
\end{align}
\end{linenomath}

ROC曲线是用来描述灵敏度的曲线,可以通过计算真阳性率和假阳性率来实现。在ROC空间中,x坐标是假阳性率值,y坐标代表真阳性率值,二者分别定义如下:
\begin{linenomath}
\begin{align}
\textrm{FalsePositiveRate =$\frac{FP}{FP+TN}$}\label{式2_40}\\
\textrm{TruePositiveRate =$\frac{TP}{TP+FN}$}\label{式2_41}
\end{align}
\end{linenomath}

在进行计算时,我们可以通过对显著图调整不同的阈值,然后与真实数据结合得到不同的PR值(或ROC值),从而可以得到一条曲线。对于PR曲线,显然查的既准又全的比较好,即越靠近坐标$(1,1)$的位置越好; 对于ROC曲线,显然真阳性率越大、假阳性率越小越好,即越靠近坐标$(0,1)$的位置越好。图~\ref{图2_13}分别给出了PR曲线和ROC曲线。
\begin{figure}[h]
  \centering%
  \begin{subfigure}{0.4\textwidth}
    \includegraphics[height=5cm]{图2_13_a}
    \caption{}
  \end{subfigure}
  \hspace{4em}%\\\\
  \begin{subfigure}{0.4\textwidth}
    \includegraphics[height=5cm]{图2_13_b}
    \caption{}
  \end{subfigure}
  \caption{PR曲线和ROC曲线图。(a)PR曲线; (b)ROC曲线}
  \label{图2_13}
\end{figure}

%-------------------------------------------------------------------------------------------------------------
\subsection{AUC值}
\label{2_5_2}

为了更好的说明ROC表达结果的好坏,AUC被提了出来,简单的说AUC就是ROC曲线下的面积\cite{Green1966book},在了解了ROC曲线的构造之后,显然AUC的数值不能大于1,而由于ROC曲线一般都处于$y=x$这条直线上,因此它的取值范围会在0.5到1之间。用AUC作评价标准的原因是大多数情况下ROC曲线并不能比较清晰的说明分类效果的好坏,而AUC作为数值则能更好的说明哪个分类器的效果更好。在显著性检测评价的情况下,人眼注视的点被看作阳性集合,从其他位置采样获得的点被看作阴性集合\cite{Tatler2005correlates}。同PR曲线和ROC曲线介绍类似,显著图可以被看作二值分类,通过阈值分割提取出阳性样本,从而得到真阳性率值和假阳性率值,画出ROC曲线,然后计算每一幅图像ROC曲线下的面积,最后通过计算整个数据集的平均值得到最终的AUC值。对于越好的注视结果预测,它所对应的AUC的值越接近1。

%-------------------------------------------------------------------------------------------------------------
\subsection{KL散度}
\label{2_5_3}

KL用来度量两个概率分布P和Q的差别,在典型情况下,P代表数据的真实分布,Q代表数据的理论分布。在显著性检测的情况下,它可以衡量注意焦点位置的视觉显著性度量直方图和随机位置的视觉显著性度量直方图之间的散度\cite{Kullback1959book,IttiCVPR2005principled}。如果一个显著性检测模型可以比较好的预测人类的注意焦点,在注意焦点的显著性度量就会大大高于随机选取的位置点,KL的值就会比较高。因此,KL的值越高,说明显著性检测模型越好。在进行计算显著行度量的直方图之间KL散度的时候,选取的两种直方图分别是场景图像中注意焦点位置的显著性脂肪图和数据库中随机选择的场景中同一位置的显著性直方图。将测试图像的显著性分为10个区间,统计出该图像所有注意焦点落在每个区间的次数,然后除以总的注视点数得到该幅图像在注意焦点处的显著性直方图,然后对测试集中的每幅图像进行这样的操作,再进行均值化处理,最后得到注意焦点处的显著性直方图。

对于随机点的选取方式,可以在同一点位置,不同的显著图中进行选取,取多次后再取均值。KL距离的公式为:
\begin{linenomath}
\begin{align}
D_{KL}=\sum_{i}P(radom(i))*log(\frac{P(random(i))}{P(fixation(i))})
\label{式2_42}
\end{align}
\end{linenomath}
公式中$i$等于显著性区间的个数,对得到的$D_{KL}$的值进行平均,从而获得最终的KL散度值。

%-------------------------------------------------------------------------------------------------------------
\subsection{线性相关系数(CC)}
\label{2_5_4}

线性相关系数在图像处理中常用来比较两幅图像的之间的关系,其广泛应用在图像配准、目标识别和视差测量中等。线性相关系数衡量变量之间的关系强度,在显著性检测中亦可以用来衡量两幅显著图之间的线性强度关系\cite{Jost2005Assessing}。线性相关系数的定义如下:
\begin{linenomath}
\begin{align}
CC(G,S)=\frac{\sum_{x,y}(G(x,y)-\mu_{G})\cdot(S(x,y)-\mu_{S})}{\sqrt{\sigma_{G}^{2}\cdot\sigma_{S}^{2}}}
\label{式2_43}
\end{align}
\end{linenomath}
其中,$\mu_{G}$和$\mu_{S}$分别是$G$和$S$的均值,$\sigma_{G}^{2}$和$\sigma_{S}^{2}$分别代表$G$和$S$的方差。选择线性相关系数的优势在于它可以通过-1到+1之间的单个标量值对比两个变量之间的相关性,当相关系数值越接近于+1或-1时,说明它们之间的相关性越强。

%-------------------------------------------------------------------------------------------------------------
\subsection{本章小节}
\label{2_5_5}

本章详细介绍了频域显著性检测方法的一般步骤,该方法具有计算简单、快速的特点。频域显著性检测算法首先进行前期处理,提取图像的底层特征,常见的是将图像进行颜色空间的转换,如转换到RGB空间、Lab空间或IRGBY空间等,然后进行图像的变换,最常见的是傅里叶变换,将图像从空间域转换到频域,还有其他的转换方法,如离散余弦变换、小波变换等,图像转换到频域后,通过分析图像对应的谱在频域的分布特点从而进行相应的频域处理,处理完再通过傅里叶反变换转换到空间域,最后通过后处理机制对显著图进行进一步增强得到最终的显著图。为了比较显著效果的好坏,我们还介绍了一些比较常用的显著性评价方法,如PR曲线、ROC曲线、AUC值和CC值等,这些评价方法保证了显著性检测方法的客观性和准确性。
