%%% Local Variables: 
%%% mode: latex
%%% TeX-master: t
%%% End: 

\chapter{总结与展望}
\label{cha5}

%=============================================================================================================
\section{论文总结}
\label{5_1}

人类可以很容易的判断图像中的显著性区域,并注意到图像的重要部分,这种机制使得视觉系统可以对有限的资源进行合理的分配。由于我们可通过显著性区域来优先分配图像分析和合成所需要的计算资源,所以通过计算来检测图像的显著性区域具有十分重要的意义。但是很多基于空间处理的显著性检测算法相对比较复杂,参数设置比较多,因此无法满足计算的速度,没有应用价值。为了提高显著图的计算速度,频域显著性检测算法可以解决这个问题。它具有简单、高效的特点,利用频域显著性检测算法可以大大提高检测的效率。因此,频域的显著性检测算法值得我们进行分析和建模。

围绕频域显著性检测机制和计算模型,本文进行了相关的研究和探索,具体可以归结为如下几个方面:
\begin{enumerate}
\item 以视觉注意为基础,介绍了视觉注意机制,研究了国内外发展的现状,并且介绍了为什么要进行频域显著性检测;
\item 详细介绍了频域显著性检测的原理和一般处理流程,包括前期处理、时频变换、频域处理、时频反变换和后期处理,对每一个过程中所所包含的类别、原理和功能进行了详细的分析;
\item 介绍了多种典型的频域显著性检测算法,包括最早提出频域处理方法的谱剩余算法、只保留相位谱的四元傅里叶变换相位谱算法、具有生物启发性的频域除法归一化显著性检测算法、基于相位谱和调谐幅度谱的显著性检测算法、基于稀疏性理论的图像签名显著性算法和基于频域尺度空间分析的显著性检查算法。
\item 总结了之前算法的优缺点,在频域尺度空间模型的基础上,本文提出了基于幅度谱分析的自适应显著目标检测算法,该算法通过显著目标尺寸与幅度谱滤波尺度的关系,自适应的消除重复的背景区域,然后通过中央偏见机制将多个显著图融合在一起形成了最后的最佳显著图。我们的方法不仅可以将显著区域均匀的显示出来,而且避免了频域尺度空间模型存在的显著信息丢失问题。最后我们通过定性定量对比实验证明了我们所提方法的有效性和优越性。
\end{enumerate}

%=============================================================================================================
\section{论文展望}
\label{5_2}

尽管本文的算法可以得到较好的结果,并且具有频域显著性检测算法简单、快速的特点,但是还存在着一些问题,需要做进一步的研究:
\begin{enumerate}
\item 我们的算法是基于Matlab平台运行的,但Matlab在计算速度上仍然不够快将来希望用C/C++语言进行编程,进一步提高算法的运行速度;
\item 将频域显著性检测与空间域显著性检测结合,从而在节省时间和提高显著性检测效果上有更大的提高;
\item 图像标签算子检测的显著目标的尺寸有时候不是很精确,下一步我们会寻找其他的方法提高显著目标尺寸的检测精度;
\item 我们采用的是自底向上的显著性检测算法,即完全靠数据驱动,将来希望将受意识支配、依赖于具体任务的自顶向下的方法加入进来,从而使检测结果更加全面,更符合人类视觉系统的处理方式。
\end{enumerate}












































